\documentclass[13pt]{beamer}
\usetheme{Frankfurt}
\usecolortheme{beaver}

\usepackage[spanish]{babel}
\usepackage{amsmath}
\usepackage{amsfonts}
\usepackage{amssymb}
\usepackage{color}
\usepackage{xcolor}
\usepackage{listings}
\usepackage{graphicx}
\usepackage{wrapfig}
\usepackage{algorithm2e}

\RestyleAlgo{ruled}
\SetKwComment{Comment}{/* }{ */}

\makeatletter
\renewcommand{\algorithmcfname}{Algoritmo}
\makeatother

\SetKwFor{While}{mientras }{}{fin}
\SetKwFor{For}{para cada }{}{fin}
\SetKwFor{If}{si}{}{fin}
%\SetKwFor{eIf}{si}{}{en otro caso}
% \SetKwFor{For}{para cada:}{}{}
%\SetKwFor{KwData}{Dato:}{}{}
%\SetKwFor{Data}{Dato:}{}{}
\SetKwFor{Results}{resultado}{}{}


\definecolor{codegreen}{rgb}{0,0.6,0}
\definecolor{codegray}{rgb}{0.5,0.5,0.5}
\definecolor{codepurple}{rgb}{0.58,0,0.82}
\definecolor{backcolour}{rgb}{0.95,0.95,0.92}

\lstdefinestyle{mystyle}{
    basicstyle=\tiny,
    commentstyle=\color{codegreen},
    keywordstyle=\color{magenta},
    numberstyle=\tiny\color{codegray},
    stringstyle=\color{codepurple},
    basicstyle=\ttfamily\footnotesize,
    breakatwhitespace=false,         
    breaklines=true,                 
    captionpos=b,                    
    keepspaces=true,                 
    numbers=left,                    
    numbersep=5pt,                  
    showspaces=false,                
    showstringspaces=false,
    showtabs=false,                  
    tabsize=2,
	xleftmargin=8pt
}

\lstset{literate=
	{á}{{\'a}}1 {é}{{\'e}}1 {í}{{\'i}}1 {ó}{{\'o}}1 {ú}{{\'u}}1
	{Á}{{\'A}}1 {É}{{\'E}}1 {Í}{{\'I}}1 {Ó}{{\'O}}1 {Ú}{{\'U}}1
	{à}{{\`a}}1 {è}{{\`e}}1 {ì}{{\`i}}1 {ò}{{\`o}}1 {ù}{{\`u}}1
	{À}{{\`A}}1 {È}{{\'E}}1 {Ì}{{\`I}}1 {Ò}{{\`O}}1 {Ù}{{\`U}}1
	{ä}{{\"a}}1 {ë}{{\"e}}1 {ï}{{\"i}}1 {ö}{{\"o}}1 {ü}{{\"u}}1
	{Ä}{{\"A}}1 {Ë}{{\"E}}1 {Ï}{{\"I}}1 {Ö}{{\"O}}1 {Ü}{{\"U}}1
	{â}{{\^a}}1 {ê}{{\^e}}1 {î}{{\^i}}1 {ô}{{\^o}}1 {û}{{\^u}}1
	{Â}{{\^A}}1 {Ê}{{\^E}}1 {Î}{{\^I}}1 {Ô}{{\^O}}1 {Û}{{\^U}}1
	{ã}{{\~a}}1 {ẽ}{{\~e}}1 {ĩ}{{\~i}}1 {õ}{{\~o}}1 {ũ}{{\~u}}1
	{Ã}{{\~A}}1 {Ẽ}{{\~E}}1 {Ĩ}{{\~I}}1 {Õ}{{\~O}}1 {Ũ}{{\~U}}1
	{œ}{{\oe}}1 {Œ}{{\OE}}1 {æ}{{\ae}}1 {Æ}{{\AE}}1 {ß}{{\ss}}1
	{ű}{{\H{u}}}1 {Ű}{{\H{U}}}1 {ő}{{\H{o}}}1 {Ő}{{\H{O}}}1
	{ç}{{\c c}}1 {Ç}{{\c C}}1 {ø}{{\o}}1 {å}{{\r a}}1 {Å}{{\r A}}1
	{€}{{\euro}}1 {£}{{\pounds}}1 {«}{{\guillemotleft}}1
	{»}{{\guillemotright}}1 {ñ}{{\~n}}1 {Ñ}{{\~N}}1 {¿}{{?`}}1 {¡}{{!`}}1 
}

\lstset{style=mystyle, basicstyle=\scriptsize}

\author{Shao Jie Hu Chen \and Mario Megías Mateo \and Jesús Samuel García Carballo}
\title{Práctica 3. Algoritmos Voraces}
\subtitle{Algorítmica}
%\logo{\includegraphics[scale=0.05]{logo-ugr.jpeg}}
\institute{Equipo Rojo}
%\date{}
%\subject{}
%\setbeamercovered{transparent}
\setbeamertemplate{navigation symbols}{}

\begin{document}
	
	\begin{frame}[plain]
		\maketitle
		% \begin{center}
		% 	\includegraphics[scale=0.15]{logo-ugr.jpeg}
		% \end{center}
	\end{frame}
	
	\begin{frame}
		\frametitle{Índice de contenidos}
		\tableofcontents
	\end{frame}

    % Introducción

    \section{Introducción}

	\begin{frame}
		\frametitle{Introducción}
		Un algoritmo greedy es aquel al que se le aplica un enfoque greedy para su resolución, es decir, el que reúne las 6 características siguientes:
		\begin{itemize}
			\item Conjunto de candidatos.
			\item Lista de candidatos ya usados.
			\item Un criterio solución, cuando se forma una solución no necesariamente óptima.
			\item Un criterio de factibilidad, candidatos que podrán llegar a ser solución.
			\item Una función de selección que indica el candidato más prometedor de los no usados.
			\item Una función objetivo que a cada solución le asocia un valor, y es la función que intentamos optimizar.
		\end{itemize}
	\end{frame}

	\begin{frame}
		\frametitle{Objetivos}
			\begin{itemize}
				\item Conocer y trabajar con algoritmos greedy, comprendiendo su enfoque.
				\item Comprender el problema del TSP, uno de los más importantes en materia de algorítmica y saber trabajar con el, para proporcionar soluciones.
				\item Estudiar la eficiencia de distintos algoritmos del problema del Viajante de Comercio.
				\item Entender y saber utilizar grafos, así como la matriz de adyacencia, ambos aplicados a la resolución de problemas.
			\end{itemize}		
	\end{frame}


    \begin{frame}
        \frametitle{Equipo empleado}

        \begin{block}{Especificaciones técnicas del equipo}
            \begin{itemize}
                \item \textbf{Procesador}: Intel(R) Core(TM) i7-9750H CPU @ 2.60GHz
                \item \textbf{Memoria RAM:} 32 GB DDR4
                \item \textbf{Sistema Operativo}: Ubuntu 20.04.4 LTS
            \end{itemize}
        \end{block}
    \end{frame}

	\section{Ejercicio 1}


    \begin{frame}
		\frametitle{Ejercicio 1}
		En este ejercicio dado un buque mercante cuya capacidad de carga es de K toneladas y un conjunto de contenedores $c_1,...,c_n$ cuyos
		pesos respectivos son $p_1,...,p_n$, debemos hallar un algoritmo que maximice el número de contenedores cargados y otro que intente 
		maximizar el número de toneladas cargadas.
	\end{frame}

    \begin{frame}
		\frametitle{Máximo número de contenedores}
		\begin{center}
			\includegraphics[scale=1.5]{./img/DibCont1.pdf}
		\end{center}
	\end{frame}

	\begin{frame}
		\frametitle{Pseudocódigo}
		\begin{algorithm}[H]
			\begin{minipage}{0.92\textwidth}
			\textbf{Parámetro}: contenedores (vector de enteros)
		
			\textbf{Parámetro}: K (capacidad del buque)
		
			\end{minipage}
		
			peso = 0\;
			vector resultado\;
			lleno = false\;
		
			sort(pesos.begin(), pesos.end());

			\For{i desde 0 hasta contenedores.tamaño-1 y !lleno} {
				\eIf{contenedores$[i]$ + peso $\leq$ K}{
					peso += contenedores$[i]$\;
				}{ lleno = true\;}
			}

			\Return{result}
			
		\end{algorithm}
	\end{frame}

	\begin{frame}
		\frametitle{Análisis de eficiencia}
		\begin{block}{Eficiencia Teórica}
			$$T(n) \in \mathcal{O}(n^2)$$
		\end{block}
	\end{frame}

	\begin{frame}
		\frametitle{Condición de optimalidad}

		\begin{lemma}
			Sea $\mathcal{S} = \{S \subset P : \sum_{p \in S} p \leq K\}$ el conjunto de soluciones. Entonces
			existe $S_m \in \mathcal S$ óptimo. 
		\end{lemma}

		\begin{lemma}
			Sea $W \in \mathcal W$ y sea $G_k$ un conjunto formado
			por la k primeros elementos escogidos mediante el algoritmo greedy. 
			Entonces se verifica $\sum_{p \in G_k} \leq \sum_{p \in W}$. 
		\end{lemma}

		\begin{theorem}
			Sean $\mathcal{S} = \{S \subset P : \sum_{p \in S} p \leq K\}$ el conjunto de soluciones y
			$P = \{p_1,p_2,\cdots,p_n\}$ el conjuntos de pesos asociados
		a los contenedores. 
			Sea $G \in \mathcal S$ la solución generada por el algoritmo greedy. Entonces
			$G$ es óptimo. 
		\end{theorem}

	\end{frame}

    \begin{frame}
		\frametitle{Máximas toneladas}
		\begin{center}
			\includegraphics[scale=1.5]{./img/DibCont2.pdf}
		\end{center}
	\end{frame}

	\begin{frame}
		\frametitle{Pseudocódigo}
		\begin{algorithm}[H]
			\begin{minipage}{0.92\textwidth}
		
			\textbf{Parámetro}: containers (vector de enteros)
		
			\textbf{Parámetro}: K (capacidad del buque)
		
			\end{minipage}
		
			weight = 0\;
			container = 0\;
			vector result\;
		  
			sort(containers.begin(),containers.end(),greater$<$int$>$());
		
			\While{$weight  \leq K$ y $container  \leq containers.size$}{      
				\If{$weight + containers.at(container) \leq K$}{
					result.push\_back(containers.at(container))\;
					$weight += containers.at(container)$\;
				}
				container++\;
			}
		  
		
			\Return{result}
		\end{algorithm}
	\end{frame}

	\begin{frame}
		\frametitle{Análisis de eficiencia}
		\begin{block}{Eficiencia Teórica}
			$$T(n) \in \mathcal{O}(n^2)$$
		\end{block}
	\end{frame}

	\section{Ejercicio 2}

    \begin{frame}
		\frametitle{Problema del Viajante del Comercio}

        \begin{itemize}
            \item Algoritmo del vecino más cercano.
            \item Algoritmo de inserción económica.
            \item Algoritmo basado en Kruskal.
        \end{itemize}

		\begin{center}
			\includegraphics[scale=0.35]{../src/PVC.png}
			\includegraphics[scale=0.35]{../src/PVC1.png}
			\includegraphics[scale=0.35]{../src/PVC2.png}
		\end{center}

	\end{frame}

    \begin{frame}
		\frametitle{Algoritmo del vecino más cercano}
		\begin{center}
			\includegraphics[scale=1.5]{./img/DibVecCercano.pdf}
		\end{center}
	\end{frame}

	\begin{frame}
		\frametitle{Pseudocódigo}

	\end{frame}

	\begin{frame}
		\frametitle{Algoritmo del vecino más cercano}
			\includegraphics[scale=0.2]{../src/vecino_ulysses.pdf}
			\includegraphics[scale=0.2]{../src/vecino_bayg.pdf}
			\begin{center}
				\includegraphics[scale=0.2]{../src/vecino_eil.pdf}
			\end{center}
	\end{frame}


    \begin{frame}
		\frametitle{Algoritmo de inserción económica}
		\begin{center}
			\includegraphics[scale=1.5]{./img/DibInsercion.pdf}
		\end{center}
	\end{frame}

	\begin{frame}
		\frametitle{Pseudocódigo}
		
	\end{frame}

	\begin{frame}
		\frametitle{Algoritmo del vecino más cercano}
			\includegraphics[scale=0.2]{../src/vecino_ulysses.pdf}
			\includegraphics[scale=0.2]{../src/insercion_bayg.pdf}
			\begin{center}
				\includegraphics[scale=0.2]{../src/insercion_eil.pdf}
			\end{center}
	\end{frame}

    \begin{frame}
		\frametitle{Algoritmo basado en Kruskal}
		\begin{center}
			\includegraphics[scale=1.5]{./img/DibPropio.pdf}
		\end{center}
	\end{frame}

	\begin{frame}
		\frametitle{Pseudocódigo}

	\end{frame}

	\begin{frame}
		\frametitle{Algoritmo del vecino más cercano}
			\includegraphics[scale=0.2]{../src/kruskal_ulysses.pdf}
			\includegraphics[scale=0.2]{../src/kruskal_bayg.pdf}
			\begin{center}
				\includegraphics[scale=0.2]{../src/kruskal_eil.pdf}
			\end{center}
	\end{frame}
	
	\begin{frame}
		\frametitle{Estudio de los tiempos}
		\begin{table}[H]
			\centering
			\begin{tabular}{|c|c|c|c|}
			  \hline
			  & bayg & eil & ulysses \\
			  \hline
			  vecino más cercano & 76 & 248 & 39 \\
			  \hline
			  inserción & 402 & 2240 & 120 \\
			  \hline
			  $\alpha$-Kruskal & 8859 & 16048 & 2559 \\
			  \hline
			\end{tabular}
			\caption{Tabla donde se indica el tiempo empleado en microsegundos en los conjuntos de prueba en función de la heurística empleada.}
		\end{table}
	\end{frame}

	\begin{frame}
		\frametitle{Análisis de eficiencia}
		
		\begin{table}[H]
			\centering
			\begin{tabular}{|c|c|c|c|}
			\hline
			& bayg & eil & ulysses \\
			\hline
			vecino más cercano & 10200 & 662 & 79 \\
			\hline
			inserción & 9607 & 575 & 70 \\
			\hline
			$\alpha$-Kruskal & 10845 & 710 & 78 \\
			\hline
			\end{tabular}
			\caption{Tabla donde se indica la distancia total del ciclo recorrido en los conjuntos de prueba
			en función de la heurística empleada.}
		\end{table}	
	\end{frame}

	\section{Conclusion}

    \begin{frame}
		\frametitle{Conclusion}
	
	\end{frame}

    % Bibliografía

    \section{Bibliografía}

    \begin{frame}
        \begin{thebibliography}{0}
            \bibitem{Verdegay2017} Verdegay Galdeano. (2017). Lecciones de Algorítmica / José Luis Verdegay. Técnica Avicam.
            \bibitem{Cormen2017} Cormen. (2017). Introduction to algorithms / Thomas H. Cormen... [et al.] (3rd ed.). PHI Learning.
            \bibitem{Garrido2018} Garrido Carrillo. (2018). Estructuras de datos avanzadas: con soluciones en C++ / A. Garrido. Universidad de Granada.  
        \end{thebibliography}
    \end{frame}

\end{document}