En este ejercicio dado un buque mercante cuya capacidad de carga es de K toneladas y un conjunto de contenedores $c_1,...,c_n$ cuyos
pesos respectivos son $p_1,...,p_n$, debemos hallar un algoritmo que maximice el número de contenedores cargados y otro que intente 
maximizar el número de toneladas cargadas.

\subsection{Maximización de contenedores}

En primer lugar, debemos ser conscientes de que la capacidad del buque debe ser menor que la suma total de los pesos 
de los contenedores, pues en caso contrario, no estaría bien definido el problema. Para resolver este problema tenemos que darnos cuenta
que lo que nos interesa es ir cogiendo aquellos contenedores cuyo peso sea más pequeño e ir introduciéndolos en el buque mientras que su suma 
sea menor que la capacidad de carga del buque, puesto que el resultado debe tener el mayor número de contenedores posible, sin importar las 
toneladas totales. 

\begin{algorithm}[H]
    \caption{Algoritmo para maximizar el número de contenedores}\label{alg:max_containers}
    \begin{minipage}{0.92\textwidth}
    \textbf{Parámetro}: containers (vector de enteros)

    \textbf{Parámetro}: K (capacidad del buque)

    \end{minipage}

    weight = 0\;
	container = 0\;
    vector result\;
  
    sort(pesos.begin(), pesos.end());

    \While{$weight <= K$ y $container <= containers.size$}{
        result.push\_back(containers.at(container))\;
        weight += containers.at(container)\;
        container++\;
    }
  
    \Return{result}
    
\end{algorithm}

Como vemos en el algoritmo la clave del proceso es antes de añadir los contenedores, ordenarlos de forma creciente para que 
en el ciclo siempre se vaya añadiendo el contenedor con menor peso.

\subsubsection{Condición de optimalidad}
Sea $S = {p_1,...,p_n}$ un conjunto de soluciones para el algoritmo y $S_g = {p_{g_1},...,p_{g_n}}$ las soluciones para el algoritmo Greedy

\subsection{Maximización de toneladas}
En este caso tenemos que maximizar las toneladas por lo que para ello una buena estrategia ir cogiendo osdjavzxnckñn