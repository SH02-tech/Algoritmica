Un algoritmo greedy es aquel al que se le aplica un enfoque greedy para su resolución, es decir, el que reúne las 6 características siguientes:
\begin{itemize}
    \item Conjunto de candidatos.
    \item Lista de candidatos ya usados.
    \item Un criterio solución, cuando se forma una solución no necesariamente óptima.
    \item Un criterio de factibilidad, candidatos que podrán llegar a ser solución.
    \item Una función de selección que indica el candidato más prometedor de los no usados.
    \item Una función objetivo que a cada solución le asocia un valor, y es la función que intentamos optimizar.
\end{itemize}

Este conjunto de algoritmos no alcanzan soluciones optimales siempre, pueden alcanzar óptimos locales, pero no los globales de los problemas.
Por eso se debe demostrar la corrección del algoritmo.