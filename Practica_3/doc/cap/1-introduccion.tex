Un algoritmo greedy es una estrategia de búsqueda al que se le aplica un enfoque 
greedy para su resolución, es decir, el que reúne las 6 características siguientes:

\begin{itemize}
    \item Conjunto de candidatos.
    \item Lista de candidatos ya usados.
    \item Un criterio solución, cuando se forma una solución no necesariamente óptima.
    \item Un criterio de factibilidad, candidatos que podrán llegar a ser solución.
    \item Una función de selección que indica el candidato más prometedor de los no usados.
    \item Una función objetivo que a cada solución le asocia un valor, y es la función que intentamos optimizar.
\end{itemize}

Este conjunto de algoritmos no alcanzan soluciones optimales siempre, pueden 
alcanzar óptimos locales, pero no los globales de los problemas.
Por eso se debe demostrar la corrección del algoritmo.

\subsection{Objetivos}

En esta práctica hemos implementado y analizado la utilidad de los algoritmos 
greedy en problemas típicos de la informática, como puede ser el problema
del viajante de comercio. En particular, nuestros objetivos para realizar
las prácticas serán las siguientes:

\begin{itemize}
    \item Conocer y trabajar con algoritmos greedy, comprendiendo su enfoque.
    \item Comprender el problema del TSP, uno de los más importantes en materia de algorítmica y saber trabajar con el, para proporcionar soluciones.
    \item Estudiar la eficiencia de distintos algoritmos del Problema del Viajante de Comercio.
    \item Entender y saber utilizar grafos, así como la matriz de adyacencia, ambos aplicados a la resolución de problemas.
\end{itemize}