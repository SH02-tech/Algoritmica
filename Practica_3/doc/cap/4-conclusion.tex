En conclusión, podemos observar que aunque el algoritmo greedy es muy eficiente, no siempre se trata del algoritmo óptimo para la resolución de un problema. 
En el primer ejercicio pudimos observar y demostrar que el algoritmo greedy es el óptimo para la resolución de ambos apartados. Sin embargo, en el ejercicio 2
vimos cómo un algoritmo greedy como el que nos inventamos, suponía unos tiempos de ejecución muy altos en comparación con los demás, lo que demuestra que no siempre
se trata del óptimo, puesto que hay algoritmos mejores.
Además nos hemos dado cuenta que existen distintos algoritmos greedy a la hora de enfocar un determinado problema y que no todos tienen por qué tener la misma eficiencia,
de hecho varía para distintas formas de atajar el problema. 