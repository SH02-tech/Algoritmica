En conclusión, podemos observar que aunque el algoritmo greedy es muy eficiente, 
no siempre se trata del algoritmo óptimo para la resolución de un problema. 
En el primer ejercicio pudimos observar y demostrar que el algoritmo greedy es 
el óptimo para la resolución de uno de los apartados. Sin embargo, en el ejercicio 2
vimos cómo un algoritmo greedy como el que nos inventamos, 
suponía unos tiempos de ejecución muy altos en comparación con los demás, lo que demuestra que no siempre
se trata del óptimo, puesto que hay algoritmos mejores.
<<<<<<< HEAD
Además nos hemos dado cuenta que existen distintos algoritmos greedy a la hora de enfocar un determinado problema y que no todos tienen por qué tener la misma eficiencia,
de hecho varía para distintas formas de atajar el problema. 
=======

Asimismo, cabe resaltar la existencia de distintos algoritmos greedy a la hora de 
enfocar un determinado problema y que no todos tienen por qué tener la misma eficiencia. 
De hecho, varía para distintas formas de atajar el problema. 

Por otra parte, también cabe observar que, si bien los algoritmos greedy no
proporcionan siempre una solución óptima, se trata de una heurística que nos
permite aproximarnos a un nivel \textbf{aceptable} a la solución que sí lo es, con la 
ventaja de que problemas que no se pueden abordar de forma óptima en tiempos
razonables, éste permite dar una solución bastante buena mediante algoritmos 
de \textbf{eficiencia polinómicos}. 
>>>>>>> 726e54372ee4a6d4dab56b8b8de720d02684ede8
