En esta sección abordaremos el problema del viajante de comercio. Dado que
se trata de un algoritmo de clase NP, es prácticamente inviable encontrar 
un algoritmo que encuentre la solución óptima en un tiempo razonable.

Por tanto, para abordar este problema se emplearán 3 heurísticas diferentes que,
si bien no siempre encuentran la solución óptima, determinan soluciones 
aceptables en tiempos razonables. Para cada una de
estas heurísticas se comparará su rendimiento en un conjunto de prueba,
midiendo los tiempos de ejecución.

\subsection{Descripción de las heurísticas}

En esta sección se realizará una descripción de cada heurística que hemos 
empleado, incluyendo tanto algoritmos, implementación como casos de ejemplo.
Dos heurísticas han sido proporcionados por los profesores: el \textit{vecino
más cercano} e \textit{inserción}, mientras que la tercera es de
elaboración propia.

\subsubsection{Vecino más cercano}

La filosofía que gobierna esta heurística es bastante sencilla. La idea 
radica en ir insertando ciudades en el conjunto de solución $S$, de manera
que para decidir la ciudad $c_{i+1}$ que se insertará se escoge aquella que
dista la menor distancia de la ciudad $c_i$ (para $c_0$ se escoge
un nodo al azar). Esta idea queda resumida en el Algoritmo~\ref{alg:vec_cercano}. 

\begin{algorithm}
    \caption{An algorithm with caption}\label{alg:vec_cercano}
    \KwData{$n \geq 0$}
    \KwResult{$y = x^n$}
    $y \gets 1$\;
    $X \gets x$\;
    $N \gets n$\;
    \While{$N \neq 0$}{
      \eIf{$N$ is even}{
        $X \gets X \times X$\;
        $N \gets \frac{N}{2}$ \Comment*[r]{This is a comment}
      }{\If{$N$ is odd}{
          $y \gets y \times X$\;
          $N \gets N - 1$\;
        }
      }
    }
\end{algorithm}

\subsubsection{Inserción económica}

En este caso para obtener la solución final se fijan $c_1,c_2,c_3$ ciudades
(siguiendo las indicaciones de la práctica, se han
escogido las ciudades situadas más al norte, este y oeste). 
Posteriormente,
para decidir la ciudad $c_{i+1}$ que se insertará en el conjunto de solución,
se elige aquella que, al insertarse en un determinado arco del ciclo 
$0 \leq j \leq i$, minimice la variación en el ciclo formado por las ciudades
$c_1,c_2,\cdots,c_{i}$. Dicha ciudad se añade intercalado entre las ciudades
$c_{j},c_{j+1}$. El algoritmo asociado a esta heurística se encuentra en 
Algoritmo~\ref{alg:insercion}. 

\begin{algorithm}[h]
    \caption{Algoritmo basado en inserción}\label{alg:insercion}
    \KwData{$C$ (vector de ciudades)}
    $Ady = MatrizAdyacencia(C)$\;
    $N = C.\text{tamaño}$\;
    $S = []$\;
    $S$.añadir($c_\alpha$), con $\alpha \in \{j \in \mathbb N : c_j.x \geq c.x, \forall c \in C\}$ \Comment*[r]{Ciudad más al este}
    $S$.añadir($c_\beta$), con $\beta \in \{j \in \mathbb N : c_j.x \leq c.x, \forall c \in C\}$ \Comment*[r]{Ciudad más al oeste}
    $S$.añadir($c_\gamma$), con $\gamma \in \{j \in \mathbb N : c_j.y \geq c.y, \forall c \in C\}$ \Comment*[r]{Ciudad más al norte}
    \While{$S.\text{tamaño} \leq N$}{
      $(i,pos) = CiudadEconomica(C, S, Ady)$\;
      $S$.insertar(pos, $c_i$)\;
      $C$.quitar($c_i$)\;
    }
\end{algorithm}

El código asociado al algoritmo Algoritmo~\ref{alg:insercion} viene especificado
a continuación.