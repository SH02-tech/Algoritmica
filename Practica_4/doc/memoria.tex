\documentclass{homework}

\usepackage[nottoc,numbib]{tocbibind}
\usepackage{amsthm}
\usepackage{titlesec}
\usepackage{caption}
\usepackage{wrapfig,lipsum,booktabs}

\usepackage{algorithm2e}
\RestyleAlgo{ruled}
\SetKwComment{Comment}{/* }{ */}

\makeatletter
\renewcommand{\algorithmcfname}{Algoritmo}
\makeatother

\SetKwFor{While}{mientras }{}{fin}
\SetKwFor{For}{para cada }{}{fin}
\SetKwFor{If}{si}{}{fin}
%\SetKwFor{eIf}{si}{}{en otro caso}
% \SetKwFor{For}{para cada:}{}{}
%\SetKwFor{KwData}{Dato:}{}{}
%\SetKwFor{Data}{Dato:}{}{}
\SetKwFor{Results}{resultado}{}{}

\titleclass{\subsubsubsection}{straight}[\subsection]

\newcounter{subsubsubsection}[subsubsection]
\renewcommand\thesubsubsubsection{\thesubsubsection.\arabic{subsubsubsection}}
\renewcommand\theparagraph{\thesubsubsubsection.\arabic{paragraph}} % optional; useful if paragraphs are to be numbered

\titleformat{\subsubsubsection}
  {\normalfont\normalsize\bfseries}{\thesubsubsubsection.}{1em}{}
\titlespacing*{\subsubsubsection}
{0pt}{3.25ex plus 1ex minus .2ex}{1.5ex plus .2ex}

\makeatletter
\renewcommand\paragraph{\@startsection{paragraph}{5}{\z@}%
  {3.25ex \@plus1ex \@minus.2ex}%
  {-1em}%
  {\normalfont\normalsize\bfseries}}
\renewcommand\subparagraph{\@startsection{subparagraph}{6}{\parindent}%
  {3.25ex \@plus1ex \@minus .2ex}%
  {-1em}%
  {\normalfont\normalsize\bfseries}}
\def\toclevel@subsubsubsection{4}
\def\toclevel@paragraph{5}
\def\toclevel@paragraph{6}
\def\l@subsubsubsection{\@dottedtocline{4}{7em}{4em}}
\def\l@paragraph{\@dottedtocline{5}{10em}{5em}}
\def\l@subparagraph{\@dottedtocline{6}{14em}{6em}}
\makeatother

\setcounter{secnumdepth}{4}
\setcounter{tocdepth}{4}


\newtheorem{theorem}{Teorema}[section]
\newtheorem{corollary}{Corolario}[theorem]
\newtheorem{lemma}[theorem]{Lema}
\newtheorem{proposition}[theorem]{Proposición}

\renewcommand{\lstlistingname}{Código}
\captionsetup[table]{name=Tabla}


\title{Práctica 4: Programación Dinámica}
\author{Shao Jie Hu Chen \\ Mario Megías Mateo \\ Jesús Samuel García Carballo}
\renewcommand{\course}{Algorítmica}
% \date{1 de abril de 2022}

\begin{document}
    \renewcommand{\refname}{Bibliografía}
    \renewcommand{\contentsname}{Índice de Contenidos}
    \renewcommand{\listtablename}{Lista de Tablas}
    
    \maketitle
    \tableofcontents
    \newpage

    \section{Introducción}
    Un algoritmo greedy es aquel al que se le aplica un enfoque greedy para su resolución, es decir, el que reúne las 6 características siguientes:
\begin{itemize}
    \item Conjunto de candidatos.
    \item Lista de candidatos ya usados.
    \item Un criterio solución, cuando se forma una solución no necesariamente óptima.
    \item Un criterio de factibilidad, candidatos que podrán llegar a ser solución.
    \item Una función de selección que indica el candidato más prometedor de los no usados.
    \item Una función objetivo que a cada solución le asocia un valor, y es la función que intentamos optimizar.
\end{itemize}

Este conjunto de algoritmos no alcanzan soluciones optimales siempre, pueden alcanzar óptimos locales, pero no los globales de los problemas.
Por eso se debe demostrar la corrección del algoritmo.
    \newpage

    \section{Ejercicio 1}
    En este ejercicio dado un buque mercante cuya capacidad de carga es de K toneladas y un conjunto de contenedores $c_1,...,c_n$ cuyos
pesos respectivos son $p_1,...,p_n$, debemos hallar un algoritmo que maximice el número de contenedores cargados y otro que intente 
maximizar el número de toneladas cargadas.

\subsection{Maximización de contenedores}

En primer lugar, debemos ser conscientes de que la capacidad del buque debe ser menor que la suma total de los pesos 
de los contenedores, pues en caso contrario, no estaría bien definido el problema. Para resolver este problema tenemos que darnos cuenta
que lo que nos interesa es ir cogiendo aquellos contenedores cuyo peso sea más pequeño e ir introduciéndolos en el buque mientras que su suma 
sea menor que la capacidad de carga del buque, puesto que el resultado debe tener el mayor número de contenedores posible, sin importar las 
toneladas totales. 

\begin{algorithm}[H]
    \caption{Algoritmo para maximizar el número de contenedores}\label{alg:max_containers}
    \begin{minipage}{0.92\textwidth}
    \textbf{Parámetro}: containers (vector de enteros)

    \textbf{Parámetro}: K (capacidad del buque)

    \end{minipage}

    weight = 0\;
	container = 0\;
    vector result\;
  
    sort(pesos.begin(), pesos.end());

    \While{$weight <= K$ y $container <= containers.size$}{
        result.push\_back(containers.at(container))\;
        weight += containers.at(container)\;
        container++\;
    }
  
    \Return{result}
    
\end{algorithm}

Como vemos en el algoritmo la clave del proceso es antes de añadir los contenedores, ordenarlos de forma creciente para que 
en el ciclo siempre se vaya añadiendo el contenedor con menor peso.

\subsubsection{Condición de optimalidad}
Sea $S = {p_1,...,p_n}$ un conjunto de soluciones para el algoritmo y $S_g = {p_{g_1},...,p_{g_n}}$ las soluciones para el algoritmo Greedy

\subsection{Maximización de toneladas}
En este caso tenemos que maximizar las toneladas por lo que para ello una buena estrategia ir cogiendo osdjavzxnckñn
    \newpage

    \section{Conclusión}
    Las principales conclusiones que hemos extraído de esta práctica son:

\begin{itemize}
	
	\item Existen problemas cuya resolución por medio de algoritmos de fuerza 
	bruta presentan una complejidad no polinomial, sin embargo con el uso de la 
	técnica de programación dinámica \textbf{podemos reducir la eficiencia a un orden 
	polinomial}. En nuestro ejemplo, \textbf{pasamos de un orden de eficiencia exponencial
	a un orden polinómico}, lo cual \textbf{supone una enorme mejora en cuanto al tiempo de 
	ejecución}.
	
	\item La técnica de programación dinámica proporciona algoritmos con órdenes de eficiencia
	polinomiales, pero esta mejora en tiempo de ejecución supone una \textbf{mayor sobrecarga en cuanto 
	a los recursos de memoria consumidos}. En nuestro ejemplo, el problema trabaja sobre dos cadenas de
	$n$ componentes, pero para la realización del algoritmo \textbf{necesitamos construir una matriz de $n^{2}$ 
	entradas, aumentando considerablemente el uso de la memoria del computador}.

	\item Además de la mejora en eficiencia respecto de algoritmos triviales, la técnica de programación
	dinámica se destaca por la \textbf{obtención de soluciones óptimas respecto a la toma de la primera decisión}.
	
	
\end{itemize}

    \begin{thebibliography}{0}
        \bibitem{Verdegay2017} Verdegay Galdeano. (2017). Lecciones de Algorítmica / José Luis Verdegay. Técnica Avicam.
        \bibitem{Cormen2017} Cormen. (2017). Introduction to algorithms / Thomas H. Cormen... [et al.] (3rd ed.). PHI Learning.
        \bibitem{Garrido2018} Garrido Carrillo. (2018). Estructuras de datos avanzadas: con soluciones en C++ / A. Garrido. Universidad de Granada.  
    \end{thebibliography}
\end{document}
