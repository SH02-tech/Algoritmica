\documentclass{homework}

\usepackage[nottoc,numbib]{tocbibind}
\usepackage{amsthm}
\usepackage{titlesec}
\usepackage{caption}
\usepackage{wrapfig,lipsum,booktabs}
\usepackage{graphicx}
\usepackage{multirow}
\graphicspath{ {img/} }

\usepackage{algorithm2e}
\RestyleAlgo{ruled}
\SetKwComment{Comment}{/* }{ */}

\makeatletter
\renewcommand{\algorithmcfname}{Algoritmo}
\makeatother

\SetKwFor{While}{mientras }{}{fin}
\SetKwFor{For}{para cada }{}{fin}
\SetKwFor{If}{si}{}{fin}
%\SetKwFor{eIf}{si}{}{en otro caso}
% \SetKwFor{For}{para cada:}{}{}
%\SetKwFor{KwData}{Dato:}{}{}
%\SetKwFor{Data}{Dato:}{}{}
\SetKwFor{Results}{resultado}{}{}

\titleclass{\subsubsubsection}{straight}[\subsection]

\newcounter{subsubsubsection}[subsubsection]
\renewcommand\thesubsubsubsection{\thesubsubsection.\arabic{subsubsubsection}}
\renewcommand\theparagraph{\thesubsubsubsection.\arabic{paragraph}} % optional; useful if paragraphs are to be numbered

\titleformat{\subsubsubsection}
  {\normalfont\normalsize\bfseries}{\thesubsubsubsection.}{1em}{}
\titlespacing*{\subsubsubsection}
{0pt}{3.25ex plus 1ex minus .2ex}{1.5ex plus .2ex}

\makeatletter
\renewcommand\paragraph{\@startsection{paragraph}{5}{\z@}%
  {3.25ex \@plus1ex \@minus.2ex}%
  {-1em}%
  {\normalfont\normalsize\bfseries}}
\renewcommand\subparagraph{\@startsection{subparagraph}{6}{\parindent}%
  {3.25ex \@plus1ex \@minus .2ex}%
  {-1em}%
  {\normalfont\normalsize\bfseries}}
\def\toclevel@subsubsubsection{4}
\def\toclevel@paragraph{5}
\def\toclevel@paragraph{6}
\def\l@subsubsubsection{\@dottedtocline{4}{7em}{4em}}
\def\l@paragraph{\@dottedtocline{5}{10em}{5em}}
\def\l@subparagraph{\@dottedtocline{6}{14em}{6em}}
\makeatother

\setcounter{secnumdepth}{4}
\setcounter{tocdepth}{4}


\newtheorem{theorem}{Teorema}[section]
\newtheorem{corollary}{Corolario}[theorem]
\newtheorem{lemma}[theorem]{Lema}
\newtheorem{proposition}[theorem]{Proposición}

\renewcommand{\lstlistingname}{Código}
\captionsetup[table]{name=Tabla}


\title{Práctica 4: Programación Dinámica}
\author{Shao Jie Hu Chen \\ Mario Megías Mateo \\ Jesús Samuel García Carballo}
\renewcommand{\course}{Algorítmica}
% \date{1 de abril de 2022}

\begin{document}
    \renewcommand{\refname}{Bibliografía}
    \renewcommand{\contentsname}{Índice de Contenidos}
    \renewcommand{\listtablename}{Lista de Tablas}
    
    \maketitle
    \tableofcontents
    \newpage

    \section{Introducción}
    En ciencias de la computación, la técnica \textbf{Divide y Vencerás} es un paradigma de diseño
algorítmico que consiste en (i) dividir un problema en pequeñas partes que sean más manejables,
(ii) resolver cada subproblema individualmente y (iii) unificar las soluciones obtenidas 
para obtener el resultado final del algoritmo original \cite{Cormen2017}. 

El \textbf{objetivo} de esta práctica es resolver una serie de problemas algorítmicos aplicando
Divide y Vencerás, comparando la resolución que podemos considerar obvia respecto a la obtenida 
aplicando esta técnica. 

\subsection{Metodología}

Para realizar esta práctica se han implementado las soluciones a cada uno de los problemas
propuestos y se ha analizado su eficiencia respecto a los algoritmos ''obvios'' para resolverlos.
Con la finalidad de \textbf{automatizar} la ejecución y la generación de datos de eficiencia
de interés, hemos empleado el mismo programa que desarrollamos en \cite{Rojo2022}.
El funcionamiento interno del \textbf{Analizador} queda ilustrado en la figura 
\ref{fig:analizador} (más información en la referencia \cite{Rojo2022}). 

\begin{figure}[h]
    \centering
    \includegraphics[scale=0.87]{img/esquema_graphkiller.pdf}
    \caption{Esquema de funcionamiento del \textbf{Analizador} \cite{Rojo2022}. Elaboración propia.}
    \label{fig:analizador}
\end{figure}

\subsection{Equipo empleado}

Para el desarrollo de esta práctica, se ha empleado un equipo de la marca HP (modelo Pavilion Gaming Laptop 15-dk0xxx) 
dotado de las siguientes prestaciones: 

\begin{itemize}
    \item \textbf{Procesador}: Intel(R) Core(TM) i7-9750H CPU @ 2.60GHz
    \item \textbf{Memoria RAM:} 32 GB DDR4
    \item \textbf{Sistema Operativo}: Ubuntu 20.04.4 LTS
\end{itemize}
    \newpage
    

    \section{Análisis de similitud de secuencias}
    \subsection{Apartado A}

\textb{Problema}: Dado un vector ordenado (de forma no decreciente) de $n$ números enteros $v$, todos 
distintos, el objetivo es determinar si existe un índice i tal que $v[i] = i$ y 
encontrarlo en ese caso. 

Para resolver el problema hemos utilizado dos algoritmos, uno básico basado en 
la búsqueda lineal, y otro basado en la técnica de Divide y Vencerás. 

El último algoritmo se ha desarrollado a partir de un análisis detallado del 
problema, a partir del cual hemos conseguido demostrar las siguientes proposiciones:

\textbf{Proposición 1.1.} Sea i con $0 \leqslant i < n$ fijo, tal que $v[i]=i+k$ con 
$k \in \mathbb N$, entonces $v[j]!=j$, $\forall j$ con $i < j < n$. 

\textit{Demostración}: Razonemos por contradicción. Supongamos que para $j > i$ se 
cumple que $v[j]=j$, entonces \exists $m \in \mathbb N$ tal que $j=i+m$. Como el 
vector está ordenado en orden no decreciente sin repetidos, como mínimo un elemento
difiere en una unidad del elemento siguiente. Podemos considerar esto último sin 
lugar a ambiguedad para todo elemento del vector, luego $v[j]=v[i]+m$. Como por hipótesis
$v[i]=i+k$, tendríamos que $v[j]=i+k+m$, pero $i=j-m$, entonces $v[j]=j-m+k+m=j+k$, lo
cual es una clara contradicción, ya que habiamos supuesto que $v[j]=j$. Por tanto, no 
existe ningún $j$ tal que $v[j]=j$.






    \newpage

    \section{Conclusión}
    Las principales conclusiones que hemos extraído de esta práctica son:

\begin{itemize}
	
	\item Existen problemas cuya resolución por medio de algoritmos de fuerza 
	bruta presentan una complejidad no polinomial, sin embargo con el uso de la 
	técnica de programación dinámica \textbf{podemos reducir la eficiencia a un orden 
	polinomial}. En nuestro ejemplo, \textbf{pasamos de un orden de eficiencia exponencial
	a un orden polinómico}, lo cual \textbf{supone una enorme mejora en cuanto al tiempo de 
	ejecución}.
	
	\item La técnica de programación dinámica proporciona algoritmos con órdenes de eficiencia
	polinomiales, pero esta mejora en tiempo de ejecución supone una \textbf{mayor sobrecarga en cuanto 
	a los recursos de memoria consumidos}. En nuestro ejemplo, el problema trabaja sobre dos cadenas de
	$n$ componentes, pero para la realización del algoritmo \textbf{necesitamos construir una matriz de $n^{2}$ 
	entradas, aumentando considerablemente el uso de la memoria del computador}.

	\item Además de la mejora en eficiencia respecto de algoritmos triviales, la técnica de programación
	dinámica se destaca por la \textbf{obtención de soluciones óptimas respecto a la toma de la primera decisión}.
	
	
\end{itemize}

    \begin{thebibliography}{0}
        \bibitem{Verdegay2017} Verdegay Galdeano. (2017). Lecciones de Algorítmica / José Luis Verdegay. Técnica Avicam.
        \bibitem{Cormen2017} Cormen. (2017). Introduction to algorithms / Thomas H. Cormen... [et al.] (3rd ed.). PHI Learning.
        \bibitem{Garrido2018} Garrido Carrillo. (2018). Estructuras de datos avanzadas: con soluciones en C++ / A. Garrido. Universidad de Granada.  
    \end{thebibliography}
\end{document}
