Las principales conclusiones que hemos extraído de esta práctica son:

\begin{itemize}
	
	\item Existen problemas cuya resolución por medio de algoritmos de fuerza 
	bruta presentan una complejidad no polinomial, sin embargo con el uso de la 
	técnica de programación dinámica \textbf{podemos reducir la eficiencia a un orden 
	polinomial}. En nuestro ejemplo, \textbf{pasamos de un orden de eficiencia exponencial
	a un orden polinómico}, lo cual \textbf{supone una enorme mejora en cuanto al tiempo de 
	ejecución}.
	
	\item La técnica de programación dinámica proporciona algoritmos con órdenes de eficiencia
	polinomiales, pero esta mejora en tiempo de ejecución supone una \textbf{mayor sobrecarga en cuanto 
	a los recursos de memoria consumidos}. En nuestro ejemplo, el problema trabaja sobre dos cadenas de
	$n$ componentes, pero para la realización del algoritmo \textbf{necesitamos construir una matriz de $n^{2}$ 
	entradas, aumentando considerablemente el uso de la memoria del computador}.

	\item Además de la mejora en eficiencia respecto de algoritmos triviales, la técnica de programación
	dinámica se destaca por la \textbf{obtención de soluciones óptimas respecto a la toma de la primera decisión}.
	
	
\end{itemize}