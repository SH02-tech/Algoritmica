
\subsection{Aplicabilidad de la programación dinámica}

Como se ha discutido previamente, para aplicar un algoritmo basado en programación
dinámica se ha de verificar las siguientes condiciones:

\begin{enumerate}
    \item Comprobación de la naturaleza n-etápica del problema. 
    \item Verificación del principio de optimalidad de Bellman. 
    \item Construcción de una ecuación recurrente. 
\end{enumerate}

\subsubsection{Naturaleza n-etápica}
En efecto, 
% para encontrar la subcadena $c_n$ más larga de longitud n, hemos de empezar 
% previamente con las subcadenas más largas de longitud 1,2... ($c_1$, $c_2$, ...). 
como primera etapa se ha de conseguir las subcadenas más largas de longitud 1, 
después obtener las subcadenas más largas de longitud 2 y, así, sucesivamente. 


\subsubsection{Verificación del principio de optimalidad de Bellman}

En esta parte, vamos a determinar que el problema que hemos expuesto verifica,
efectivamente, el principio de optimalidad de Bellman. Para ello, veamos que el
problema verifica una subestructura optimal.

\textbf{Notación}. Sea $X$ una secuencia de caracteres. Definimos prefijo i-ésimo
de X, notado por $X_i$, al vector formado por los i primeros elementos de $X$. 

\begin{theorem}
    
\end{theorem}