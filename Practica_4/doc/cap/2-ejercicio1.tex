\subsection{Enunciado}
Dos hermanos fueron separados al nacer y mediante un programa de televisión se han
enterado que podrían ser hermanos. Ante esto, los dos están de acuerdo en hacerse un test de
ADN para verificar si realmente son hermanos. Se debe encontrar el porcentaje de similitud que existe 
entre estos posibles hermanos, $($como es un ejemplo lo haremos para 2 entradas posibles$)$.

\subsection{Metodología}
Para llevar a cabo este algoritmo mediante programación dinámica vamos a construir la matriz de 
ocurrencias. Para ello vamos a construir una función que la calcule y luego la implementaremos.

Sean $n,m \in \mathbb{N}$, dadas dos secuencias $X = { x_1,x_2,...,x_m}$ e $Y = { y_1,y_2,...,y_n}$, llamaremos $L(i,j)$ a la 
longitud de la secuencia común máxima de las secuencias $X_i = {x_1,...,x_i}$ e $Y_j = {y_1,...,y_j}$ $\forall i \in {1,...,n} $ y $\forall i \in {1,...,m}$, 
que se define como:  

\[
  L(i,j) = 
  \left \{
    \begin{aligned}
      0 &,\ \text{si} \ i = 0 o j = 0\\
      L(i,j) + 1 &,\ \text{si} \ i \neq  1 , j \neq  0  y x_i = y_j\\
      max(L(i,j-1) , L(i-1,j))&,\ \text{si} \ i \neq 1 , j \neq 0 y x_i \neq y_j
    \end{aligned}
  \right .
\]

En esta matriz tenemos que hacer que cada elemento de la cadena X sea una posición de cada columna 
y cada elemento de la cadena Y ocupe una posición de cada fila. Con esta disposición, vamos recorriendo
la matriz por filas y si fijado el elemento de cada fila encontramos un elemento igual que esté en una 
posición de las columnas, incrementamos la longitud.

\subsection{Aplicabilidad de la programación dinámica}

Como se ha discutido previamente, para aplicar un algoritmo basado en programación
dinámica se ha de verificar las siguientes condiciones:

\begin{enumerate}
    \item Comprobación de la naturaleza n-etápica del problema. 
    \item Verificación del principio de optimalidad de Bellman. 
    \item Construcción de una ecuación recurrente. 
\end{enumerate}

\subsubsection{Naturaleza n-etápica}
En efecto, 
% para encontrar la subcadena $c_n$ más larga de longitud n, hemos de empezar 
% previamente con las subcadenas más largas de longitud 1,2... ($c_1$, $c_2$, ...). 
como primera etapa se ha de conseguir las subcadenas más largas de longitud 1, 
después obtener las subcadenas más largas de longitud 2 y, así, sucesivamente. 


\subsubsection{Verificación del principio de optimalidad de Bellman}

En esta parte, vamos a determinar que el problema que hemos expuesto verifica,
efectivamente, el principio de optimalidad de Bellman. Para ello, veamos que el
problema verifica una subestructura optimal.

\textbf{Notación}. Sea $X$ una secuencia de caracteres. Definimos prefijo i-ésimo
de X, notado por $X_i$, al vector formado por los i primeros elementos de $X$. 

\begin{theorem}
    
\end{theorem}
% Pag 392


\section{Análisis de eficiencia teórico}

A continuación, analizaremos la eficiencia del algoritmo en notación asintótica $BigO$.
Llamamos $n$ a la longitud de las secuencias (ambas tienen el mismo número de componentes). En primer lugar nos encontramos con un doble ciclo for que recorre ambas
secuencias y contiene operaciones $O(1)$, luego la eficiencia del bloque es $O(n^{2})$.
Posteriormente tenemos sentencias de asignación que serán despreciadas, y llegamos a un
ciclo while que realiza $n$ iteraciones, en cuyo interior tenemos operaciones elementales que son $O(1)$, y finalmente obtemos la subsecuencia con otro bucle while, que puede realizar un número inferior o igual a $n$ operaciones. Por tanto, aplicando la regla del máximo, obtenemos que \textbf{nuestro algoritmo tiene una eficiencia de $O(n^{2})$}.
 
















