\subsection{Enunciado}
Dos hermanos fueron separados al nacer y mediante un programa de televisión se han
enterado que podrían ser hermanos. Ante esto, los dos están de acuerdo en hacerse un test de
ADN para verificar si realmente son hermanos. Se debe encontrar el porcentaje de similitud que existe 
entre estos posibles hermanos, $($como es un ejemplo lo haremos para 2 entradas posibles$)$.

\subsection{Metodología}
Para llevar a cabo este algoritmo mediante programación dinámica vamos a construir la matriz de 
ocurrencias. Para ello vamos a construir una función que la calcule y luego la implementaremos.

Sean $n,m \in \mathbb{N}$, dadas dos secuencias $X = { x_1,x_2,...,x_m}$ e $Y = { y_1,y_2,...,y_n}$, llamaremos $L(i,j)$ a la 
longitud de la secuencia común máxima de las secuencias $X_i = {x_1,...,x_i}$ e $Y_j = {y_1,...,y_j}$ $\forall i \in {1,...,n} $ y $\forall i \in {1,...,m}$, 
que se define como:  

\[
  L(i,j) = 
  \left \{
    \begin{aligned}
      0 &,\ \text{si} \ i = 0 o j = 0\\
      L(i,j) + 1 &,\ \text{si} \ i \neq  1 , j \neq  0  y x_i = y_j\\
      max(L(i,j-1) , L(i-1,j))&,\ \text{si} \ i \neq 1 , j \neq 0 y x_i \neq y_j
    \end{aligned}
  \right .
\]

En esta matriz tenemos que hacer que cada elemento de la cadena X sea una posición de cada columna 
y cada elemento de la cadena Y ocupe una posición de cada fila. Con esta disposición, vamos recorriendo
la matriz por filas y si fijado el elemento de cada fila encontramos un elemento igual que esté en una 
posición de las columnas, incrementamos la longitud.

\subsection{Aplicabilidad de la programación dinámica}

Como se ha discutido previamente, para aplicar un algoritmo basado en programación
dinámica se ha de verificar las siguientes condiciones:

\begin{enumerate}
    \item Comprobación de la naturaleza n-etápica del problema. 
    \item Verificación del principio de optimalidad de Bellman. 
    \item Construcción de una ecuación recurrente. 
\end{enumerate}

\subsubsection{Naturaleza n-etápica}
En efecto, 
% para encontrar la subsecuencia $c_n$ más larga de longitud n, hemos de empezar 
% previamente con las subsecuencias más largas de longitud 1,2... ($c_1$, $c_2$, ...). 
como primera etapa se ha de conseguir las subsecuencias más largas de longitud 1, 
después obtener las subsecuencias más largas de longitud 2 y, así, sucesivamente. 


\subsubsection{Verificación del principio de optimalidad de Bellman}

En esta parte, vamos a determinar que el problema que hemos expuesto verifica,
efectivamente, el principio de optimalidad de Bellman. Para ello, veamos que el
problema verifica una subestructura optimal. El método y el teorema se pueden encontrar
en \cite{Cormen2017}. 

\textbf{Notación}. Sea $X$ una secuencia de caracteres. Definimos \textbf{prefijo i-ésimo}
de X, notado por $X_i$, al vector formado por los i primeros elementos de $X$. 
También definimos por \textbf{subsecuencia común} a dos secuencias de caracteres $X,Y$ a cualquier sucesión de caracteres
cuyo orden ascendente se encuentre tanto en $X$ como en $Y$. Además, diremos que la
subsecuencia común es \textbf{máxima} cuando su número de caracteres sea igual o superior 
a cualquier otra subsecuencia común. 

\begin{theorem}
    Sean $X=(x_1,x_2,\cdots, x_m),Y=(y_1,y_2, \cdots, y_n)$ secuencias de caracteres 
    y sea $Z$ cualquier subsecuencia común máxima 
    de $X$ e $Y$. Entonces:
    \begin{enumerate}
      \item Si $x_m = y_n$, entonces $z_k = x_m = y_n$ y $Z_{k-1}$ es una subsecuencia
      común máxima de $X_{m-1}$ e $Y_{n-1}$. 
      \item Si $x_m \neq y_n$, entonces:
      \begin{enumerate}
        \item $z_k \neq x_m$ implica que $Z$ es una subsecuencia común máxima para $X_{m-1}$ e $Y$. 
        \item $z_k \neq y_n$ implica que $Z$ es una subsecuencia común máxima para $X$ e $Y_{n-1}$. 
      \end{enumerate}
    \end{enumerate}
\end{theorem}
% Pag 392

\begin{proof}
  Empezamos probando (1). Supongamos que $x_m = y_n$. Si fuese $z_k \neq x_m$,
  podríamos añadir $x_m = y_n$ para obtener una subsecuencia común de $X$ e $Y$ 
  de longitud k+1, contradiciendo la hipótesis de subsecuencia común máxima de Z. 
  Por tanto, ha de ser $z_k = x_m = y_n$. Ahora, el prefijo $Z_{k-1}$ es una
  subsecuencia común de k-1 caracteres, subsecuencia común de $X_{m-1}$ e $Y_{n-1}$.
  Veamos que se trata de una subsecuencia común máxima. Para ello, procedemos
  nuevamente por reducción al absurdo. Supongamos que existe una subsecuencia común $W$ de
  $X_{m-1}$ e $Y_{n-1}$ con longitud mayor que $k-1$. Entonces, añadiendo $x_m$ a $W$
  se produce una subsecuencia común de $X$ e $Y$ cuya longitud es mayor que k, lo
  que sería contradictorio.

  Para (2), empezamos probando el primero. Si $z_k \neq x_m$, entonces $Z$ es una
  subsecuencia común de $X_{m-1}$ e $Y$. Si hubiese una subsecuencia común $W$
  de longitud mayor que $k$, entonces $W$ sería nuevamente una subsecuencia común 
  de $X_m$ e $Y$, contradiciendo que $Z$ sea una subsecuencia común máxima.
  Análogamente se haría para el caso $z_k \neq y_n$. 
\end{proof}

Este teorema demuestra que el problema verifica el principio de optimalidad de
Bellman, pues cualquier política óptima únicamente puede estar formado por
subpolíticas óptimas. 

\subsubsection{Construcción de la ecuación recurrente}

