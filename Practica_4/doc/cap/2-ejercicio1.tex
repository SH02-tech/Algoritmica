
\subsection{Aplicabilidad de la programación dinámica}

Como se ha discutido previamente, para aplicar un algoritmo basado en programación
dinámica se ha de verificar las siguientes condiciones:

\begin{enumerate}
    \item \textbf{Comprobación de la naturaleza n-etápica del problema}. En efecto, 
    para encontrar la subcadena $c_n$ más larga de longitud n, hemos de empezar 
    previamente con las subcadenas más largas de longitud 1,2... ($c_1$, $c_2$, ...). 

    \item \textbf{Verificación del principio de optimalidad de Bellman}. Para ello, 
    veamos que 
    \item Construcción de una ecuación recurrente. 
\end{enumerate}



% \subsection{Verificación del principio de optimalidad de Bellman}

% En esta parte, vamos a determinar que el problema que hemos expuesto verifica,
% efectivamente, el principio de optimalidad de Bellman. Para ello, consideraremos
% dos cadenas de longitud $m,n \in \mathbb N$. Lo que vamos a considerar es 
% fijar $m \in \mathbb N$ y variar $n$ mediante un proceso inductivo.

% Pag 392