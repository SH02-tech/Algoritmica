<<<<<<< HEAD
\subsection{Enunciado}
Dos hermanos fueron separados al nacer y mediante un programa de televisión se han
enterado que podrían ser hermanos. Ante esto, los dos están de acuerdo en hacerse un test de
ADN para verificar si realmente son hermanos. Se debe encontrar el porcentaje de similitud que existe 
entre estos posibles hermanos, $($como es un ejemplo lo haremos para 2 entradas posibles$)$.

\subsection{Metodología}
Para llevar a cabo este algoritmo mediante programación dinámica vamos a construir la matriz de 
ocurrencias. Para ello vamos a construir una función que la calcule y luego la implementaremos.

Sean $n,m \in \mathbb{N}$, dadas dos secuencias $X = { x_1,x_2,...,x_m}$ e $Y = { y_1,y_2,...,y_n}$, llamaremos $L(i,j)$ a la 
longitud de la secuencia común máxima de las secuencias $X_i = {x_1,...,x_i}$ e $Y_j = {y_1,...,y_j}$ $\forall i \in {1,...,n} $ y $\forall i \in {1,...,m}$, 
que se define como:  

\[
  L(i,j) = 
  \left \{
    \begin{aligned}
      0 &,\ \text{si} \ i = 0 o j = 0\\
      L(i,j) + 1 &,\ \text{si} \ i \neq  1 , j \neq  0  y x_i = y_j\\
      max(L(i,j-1) , L(i-1,j))&,\ \text{si} \ i \neq 1 , j \neq 0 y x_i \neq y_j
    \end{aligned}
  \right .
\]

En esta matriz tenemos que hacer que cada elemento de la cadena X sea una posición de cada columna 
y cada elemento de la cadena Y ocupe una posición de cada fila. Con esta disposición, vamos recorriendo
la matriz por filas y si fijado el elemento de cada fila encontramos un elemento igual que esté en una 
posición de las columnas, incrementamos la longitud.
=======

\subsection{Aplicabilidad de la programación dinámica}

Como se ha discutido previamente, para aplicar un algoritmo basado en programación
dinámica se ha de verificar las siguientes condiciones:

\begin{enumerate}
    \item \textbf{Comprobación de la naturaleza n-etápica del problema}. En efecto, 
    para encontrar la subcadena $c_n$ más larga de longitud n, hemos de empezar 
    previamente con las subcadenas más largas de longitud 1,2... ($c_1$, $c_2$, ...). 

    \item \textbf{Verificación del principio de optimalidad de Bellman}. Para ello, 
    veamos que 
    \item Construcción de una ecuación recurrente. 
\end{enumerate}



% \subsection{Verificación del principio de optimalidad de Bellman}

% En esta parte, vamos a determinar que el problema que hemos expuesto verifica,
% efectivamente, el principio de optimalidad de Bellman. Para ello, consideraremos
% dos cadenas de longitud $m,n \in \mathbb N$. Lo que vamos a considerar es 
% fijar $m \in \mathbb N$ y variar $n$ mediante un proceso inductivo.

% Pag 392
>>>>>>> b57cbcb5bdeb3918da5b756aae48124a3966337a
