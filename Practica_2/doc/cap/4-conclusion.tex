En este trabajo hemos implementado diversos algoritmos aplicando
la técnica Divide y Vencerás, y lo hemos enfrentado frente a soluciones aplicando
algoritmos clásicos mediante un análisis de eficiencia tanto teórico como empírico 
(incluso híbrido). 

Tras analizar los datos obtenidos de cada situación que se nos ha presentado,
hemos \textbf{extraído las siguientes conclusiones: }

\begin{itemize}
    \item La determinación del umbral óptimo es muy importante para garantizar un buen funcionamiento
    de la técnica Divide y Vencerás, \textbf{evitando un exceso de llamadas recursivas}.
    \item El número de repeticiones de un algoritmo para la obtención de datos \textbf{depende del tamaño del 
    del problema}, teniendo en cuenta la \textbf{casuística} de los datos.  
    \item En los equipos empleados, el orden de eficiencia es un factor limitante del mayor tamaño de 
    problema que puede llegar a ejecutase para un algoritmo determinado.
    \item La técnica Divide y Vencerás \textbf{no siempre proporciona una mejora de eficiencia}
    respecto al caso obvio. 
    
\end{itemize}