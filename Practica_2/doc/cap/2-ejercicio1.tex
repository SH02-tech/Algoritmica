El \textbf{problema} que se nos plantea es, dado un vector ordenado (de forma no decreciente) de $n$ números enteros $v$, todos 
distintos, determinar si existe un índice $i$ tal que $v[i] = i$ y 
encontrarlo en ese caso. Para resolver el problema hemos utilizado dos algoritmos, uno básico basado en 
la búsqueda lineal, y otro basado en la técnica de Divide y Vencerás. 

\subsection{Vector con valores no repetidos}

En primer lugar, abordamos el problema para vectores con valores no repetidos. Para esta parte
caben destacar los lemas \ref{lem:1} y \ref{lem:2}. En esta subsección, $v$ será un vector ordenado con
valores no repetidos con componentes enteras. 

\begin{lemma}
    \label{lem:1}
    Sea i con $0 \leqslant i < n$ fijo, tal que $v[i]=i+k$, con $k \in \mathbb N$. 
    Entonces $v[j] \neq j$, $\forall j \in \mathbb N$ con $i < j < n$. 
\end{lemma}

\begin{proof}
    Razonemos por contradicción. Supongamos que para $j > i$ se 
    cumple que $v[j]=j$, esto es, $\exists m \in \mathbb N$ tal que $j=i+m$. Como el 
    vector está ordenado en orden no decreciente sin repetidos, como mínimo un elemento
    difiere en una unidad del elemento siguiente. En consecuencia,  $v[j] \geq v[i]+m$. 
    Como por hipótesis $v[i]=i+k$, tendríamos que $v[j] \geq i+k+m$, pero $i=j-m$, entonces 
    $v[j] \geq j-m+k+m=j+k > j$, lo que es contradictorio pues habíamos supuesto que $v[j]=j$. Por tanto, no 
    existe ningún $j > i$ tal que $v[j]=j$.
\end{proof}

\begin{lemma}
    \label{lem:2}
    Sea i con $0 < i < n$ fijo, tal que $v[i]=i-k$ con 
    $k \in \mathbb N$. Entonces $v[j] \neq j$,  $\forall j$ con $0 \leqslant j < i$. 
\end{lemma}

\begin{proof}
    Razonemos por contradicción. Supongamos que para $j < i$ se 
    cumple que $v[j]=j$, entonces $\exists m \in \mathbb N$ tal que $j=i-m$. Como el 
    vector está ordenado en orden no decreciente sin repetidos, como mínimo un elemento
    difiere en una unidad del elemento siguiente. En consecuencia,  $v[j] \leq v[i]-m$. 
    Como por hipótesis $v[i]=i-k$, tendríamos que $v[j] \leq i-k-m$, pero $i=j+m$, entonces 
    $v[j] \leq j+m-k-m=j-k < j$, lo que es contradictorio pues habíamos supuesto que $v[j]=j$. Por tanto, no 
    existe ningún $j<i$ tal que $v[j]=j$.
\end{proof}

\subsubsection{Caso obvio} \label{sec:1a-obvio}

En este caso el algoritmo de búsqueda del índice se basa en la búsqueda lineal. Un ejemplo de \textbf{pseudocódigo} 
se puede encontrar en \ref{alg:1a-obvio} y una \textbf{implementación} en \ref{cod:1a-obvio}.

\subsubsubsection{Pseudocódigo}

\lstinputlisting[caption=Pseudocódigo asociado al caso obvio., label={alg:1a-obvio}]{listing/ejer1a-pseudo-obvio.txt}

\subsubsubsection{Implementación}

\lstinputlisting[language=C++, firstline=85, lastline=106, caption=Implementación en C++ 
del algoritmo basado en la búsqueda lineal., label={cod:1a-obvio}]{../src/ejercicio-1-comp-fija-no-repetidos.cpp} 

\subsubsubsection{Análisis de Eficiencia}

Las dos primeras líneas son sentencias de asignación $O(1)$, luego podemos acotarlas por una constante $c$. 
Posteriormente, nos encontramos con un bucle while, que en el peor de los casos (no se encuentra la posición buscada) 
recorre todas las componentes del vector, en cuyo cuerpo encontramos sentencias $O(1)$, y las acotamos por una constante $t$.
Entonces llamando $n$ a la longitud del vector donde realizamos la búsqueda (el tamaño del problema) tendríamos: 

\begin{equation}
    T(n) = \sum_{i=0}^{n-1} t + c = t \sum_{i=0}^{n-1} 1 + c = tn + c \implies \boxed{T(n) \in O(n)}
\end{equation}

En cuanto al análisis empírico, obtenemos:

\begin{equation}
\boxed{f(x) = 3,3768 \cdot 10^{-6} x + 0,9999, R^2 = 0.9997}
\label{eq:1a-obv-regresion}
\end{equation}

\begin{table}
	\footnotesize
	\centering
	\begin{tabular}{|r|r|}
		\hline
		$N$ & $T (ms)$ \\
		\hline
		1 & 0.0010 \\ 
		250000 & 0.8788 \\ 
		500000 & 1.7825 \\ 
		750000 & 2.6881 \\ 
		1000000 & 3.5530 \\ 
		1250000 & 4.4430 \\ 
		1500000 & 5.3485 \\ 
		1750000 & 6.2605 \\ 
		2000000 & 7.1393 \\ 
		2250000 & 8.0418 \\ 
		2500000 & 8.9705 \\ 
		2750000 & 9.8276 \\ 
		3000000 & 10.7256 \\ 
		3250000 & 11.6909 \\ 
		3500000 & 12.5136 \\ 
		3750000 & 13.5292 \\ 
		4000000 & 14.3098 \\ 
		4250000 & 15.2242 \\ 
		4500000 & 16.1069 \\ 
		4750000 & 17.0827 \\ 
		5000000 & 17.8853 \\ 
		5250000 & 18.9099 \\ 
		5500000 & 19.7304 \\ 
		5750000 & 20.5713 \\ 
		6000000 & 21.7051 \\ 
		6250000 & 22.5353 \\ 
		6500000 & 23.2494 \\ 
		6750000 & 24.2972 \\ 
		7000000 & 25.2225 \\ 
		7250000 & 26.2071 \\ 
		7500000 & 27.1841 \\ 
		7750000 & 27.7329 \\ 
		8000000 & 28.8121 \\ 
		8250000 & 29.5123 \\ 
		8500000 & 30.5612 \\ 
		8750000 & 31.4096 \\ 
		9000000 & 32.1943 \\ 
		9250000 & 33.1875 \\ 
		9500000 & 34.1360 \\ 
		9750000 & 35.9863 \\
		\hline
	\end{tabular}
	
	\caption{Tiempos de ejecución para el algoritmo \ref{alg:1a-obvio}, implementación \ref{cod:1a-obvio}.}
	\label{tab:1a-obvio}
\end{table}

\begin{figure}
	\centering
	\includegraphics[scale=0.76]{img/e1a-obv}
	\caption{Gráfica de tiempos de ejecución para el caso obvio, 
		con f(x) y coeficiente de regresión especificada por \ref{eq:1a-obv-regresion}.}
	\label{fig:1a-obv-graph}
\end{figure}

\newpage

\subsubsection{Caso Divide y Vencerás} \label{sec:1a-dyv}

En esta parte, como consecuencia de los lemas \ref{lem:1} y \ref{lem:2}, podemos emplear un 
algoritmo similar al de búsqueda binaria (un caso muy particular de Divide y Vencerás: la
\textbf{simplificación} \cite{Verdegay2017}). Un ejemplo de \textbf{pseudocódigo} se puede encontrar en \ref{alg:1a-dyv} y
una \textbf{implementación} en \ref{cod:1a-dyv}.

\subsubsubsection{Pseudocódigo}

\lstinputlisting[caption=Pseudocódigo asociado al caso Divide y Vencerás., label={alg:1a-dyv}]{listing/ejer1a-pseudo-dyv.txt}

\subsubsubsection{Implementación}

\lstinputlisting[language=C++, firstline=111, lastline=132, 
caption=Implementación en C++ del algoritmo basado en Divide y Vencerás., label={cod:1a-dyv}]
{../src/ejercicio-1-comp-fija-no-repetidos.cpp} 

\subsubsubsection{Determinación del umbral}

Un algoritmo Divide y Vencerás debe evitar proceder de forma recursiva cuando el tamaño de los subcasos no lo justifique, ya que
de lo contrario su rendimiento se degrada considerablemente. 

Para encontrar dicho umbral, hemos de determinar el número de datos óptimo tales que para valores menores que este, el algoritmo básico 
presente mejores tiempos de ejecución que el algoritmo Divide y Vencerás. El cálculo del umbral depende en gran medida de las constantes ocultas, 
luego es dificil hacer un cáculo fiel a nivel teórico, por lo que se opta por adoptar un punto de vista teórico/empírico, mediante el cual, a partir de las 
implementaciones ejecutadas en un equipo concreto, \textbf{obtenemos las funciones de ajuste de ambos algoritmos con sus correspondientes 
parámetros}, y una vez obtenidas, \textbf{calculamos el umbral de forma teórica igualando ambas funciones, despejando el tamaño del problema}: 

\newpage

\begin{table}
	\footnotesize
	\centering
	\begin{tabular}{|r|r|r|}
		\hline
		$N$ & $T_{DyV} (ms)$ & $T_{lin} (ms)$ \\
		\hline
		1 & 0.0052 & 0.0010 \\ 
		250000 & 7.2312 & 0.8788 \\ 
		500000 & 7.6409 & 1.7825 \\ 
		750000 & 7.9035 & 2.6881 \\ 
		1000000 & 8.0444 & 3.5530 \\ 
		1250000 & 8.1743 & 4.4430 \\ 
		1500000 & 8.2631 & 5.3485 \\ 
		1750000 & 8.3702 & 6.2605 \\ 
		2000000 & 8.4370 & 7.1393 \\ 
		2250000 & 8.5165 & 8.0418 \\ 
		2500000 & 8.5908 & 8.9705 \\ 
		2750000 & 8.6333 & 9.8276 \\ 
		3000000 & 8.6589 & 10.7256 \\ 
		3250000 & 8.7636 & 11.6909 \\ 
		3500000 & 8.7922 & 12.5136 \\ 
		3750000 & 8.8138 & 13.5292 \\ 
		4000000 & 8.8514 & 14.3098 \\ 
		4250000 & 8.8867 & 15.2242 \\ 
		4500000 & 8.9199 & 16.1069 \\ 
		4750000 & 8.9514 & 17.0827 \\ 
		5000000 & 8.9813 & 17.8853 \\ 
		5250000 & 9.0097 & 18.9099 \\ 
		5500000 & 9.0367 & 19.7304 \\ 
		5750000 & 9.0626 & 20.5713 \\ 
		6000000 & 9.0737 & 21.7051 \\ 
		6250000 & 9.0812 & 22.5353 \\ 
		6500000 & 9.1306 & 23.2494 \\ 
		6750000 & 9.1560 & 24.2972 \\ 
		7000000 & 9.1771 & 25.2225 \\ 
		7250000 & 9.1927 & 26.2071 \\ 
		7500000 & 9.2173 & 27.1841 \\ 
		7750000 & 9.2335 & 27.7329 \\ 
		8000000 & 9.2549 & 28.8121 \\ 
		8250000 & 9.2542 & 29.5123 \\ 
		8500000 & 9.2936 & 30.5612 \\ 
		8750000 & 9.3070 & 31.4096 \\ 
		9000000 & 9.3145 & 32.1943 \\ 
		9250000 & 9.3394 & 33.1875 \\ 
		9500000 & 9.3354 & 34.1360 \\ 
		9750000 & 9.4071 & 35.9863 \\ 
		\hline
	\end{tabular}

	\caption{Comparación de los tiempos de ejecución para los algoritmos con implementaciones \ref{cod:1a-dyv} y \ref{cod:1a-obvio}}
	\label{tab:1a-com}
\end{table}

\begin{figure}
	\centering
	\includegraphics[scale=0.76]{img/e1a-umbral.pdf}
	\caption{Gráfica de tiempos de ejecución para el caso obvio y Divide y Vencerás, para apreciar el umbral
	óptimo.}
	\label{fig:1a-com-graph}
\end{figure}

Una primera aproximación puede realizarse mediante la representación gráfica \ref{fig:1a-com-graph}. Por tanto, podemos
determinar que el umbral es el especificado en la ecuación \ref{eq:1a-umbral}. Por tanto, podemos considerar como intervalo
umbral $[5,6]$. 

\begin{equation}
	UMBRAL \approx 6
	\label{eq:1a-umbral}
\end{equation}

\subsubsubsection{Análisis de Eficiencia}

Tenemos una estructura condicional. Si el tamaño del problema (número de componentes del vector donde 
vamos a buscar) es menor o igual que el UMBRAL escogido, entonces ejecutamos la BUSQUEDA\_LINEAL, cuyo orden de eficiencia es \ref{eq:1a-obvio-eficiencia}. 
En caso contrario, calculamos $k$, proceso que es $O(1)$ y se acota por una contante $c$. 

\begin{equation}
	\boxed{T(n) \in O (n)} \text{ (caso obvio)}
	\label{eq:1a-obvio-eficiencia}
\end{equation}

A continuación, comprobamos si hemos encontrado el elemento o no. Como estamos realizando un análisis asintótico, supondremos
el peor de los casos, en el que no encontramos el elemento, por tanto, como se ha mencionado anteriormente,
se efectúa una simplificación realizándose una llamada recursiva para la mitad del vector correspondiente.

En conclusión, el análisis quedaría (supondremos que $n$ es potencia de 2, ya que podemos realizar una acotación 
superior de $n$ por la potencia de dos más cercana al estudiar la eficiencia $Big O$): 

\begin{equation}
    T(n) = \left\{ \begin{array}{lr} T(n/2) + c & \text{si } n > \text{UMBRAL}\\ n & \text{si } n \leqslant \text{UMBRAL} \end{array} \right.
    \label{eq:1a-efi-dyv-rec}
\end{equation}

Vamos a resolver la recurrencia:

\begin{equation}
    \begin{array}{lr}  T(n) =  T(n/2) + c & \text{si } n > \text{UMBRAL} \end{array}
    \label{eq:ejer1:efi-dyv}
\end{equation}

Realizamos el cambio de variable $n = 2^{m}$ en la ecuación \ref{eq:ejer1:efi-dyv} y obtenemos:

\begin{equation*}
    T(2^{m}) =  T(2^{m-1}) + c 
\end{equation*}

Renombramos la expresión anterior como $T(2^{m}) = t_{m}$, de donde obtenemos: $t_{m} - t_{m-1} = c$, con
ecuación asociada $(x-1)^{2} = 0$. Por tanto las soluciones son de la forma: 

\begin{equation*}
    t_{m} = c_{1} + c_{2}m
\end{equation*}

Deshacemos el cambio de variable:

\begin{equation}
    T(n) = c_{1} + c_{2} \log_2(n) \implies \boxed{T(n) \in O(\log_2(n))}
    \text{ (caso Divide y Vencerás)}
    \label{eq:1a-eficiencia-lineal}
\end{equation}

En este caso, para tomar las medidas experimentales, hemos tomado para valores pequeños de $n$ una 
\textbf{media de 1.000.000 de vectores diferentes}. Esto se debe a que el número de operaciones del 
algoritmo fluctúa mucho y puede terminar mucho antes o después en función
de los valores de entrada. \textbf{Para determinar este valor}, hemos probado con diferente número de entradas, 
viendo que para este valor las salidas del tiempo de ejecución son más o menos \textbf{estables}. 

Para los análisis de eficiencia \textbf{empírico e híbrido}, se han obtenido las \textbf{funciones de ajuste}
y \textbf{coeficiente de regresión} especificadas por \ref{eq:1a-dyv-regresion}. Una representación
gráfica de los datos obtenidos, así como de la función de regresión, se puede encontrar en \ref{fig:1a-dyv-graph}.

\begin{equation}
    \boxed{f(x) = 0,5832 \log(x) + 0,021, R^2 = 0,9988}
    \label{eq:1a-dyv-regresion}
\end{equation}

\begin{table}
	\footnotesize
	\centering
	\begin{tabular}{|r|r|}
		\hline
		$N$ & $T (ms)$ \\
		\hline
		1 & 0.0052 \\ 
		250000 & 7.2312 \\ 
		500000 & 7.6409 \\ 
		750000 & 7.9035 \\ 
		1000000 & 8.0444 \\ 
		1250000 & 8.1743 \\ 
		1500000 & 8.2631 \\ 
		1750000 & 8.3702 \\ 
		2000000 & 8.4370 \\ 
		2250000 & 8.5165 \\ 
		2500000 & 8.5908 \\ 
		2750000 & 8.6333 \\ 
		3000000 & 8.6589 \\ 
		3250000 & 8.7636 \\ 
		3500000 & 8.7922 \\ 
		3750000 & 8.8138 \\ 
		4000000 & 8.8514 \\ 
		4250000 & 8.8867 \\ 
		4500000 & 8.9199 \\ 
		4750000 & 8.9514 \\ 
		5000000 & 8.9813 \\ 
		5250000 & 9.0097 \\ 
		5500000 & 9.0367 \\ 
		5750000 & 9.0626 \\ 
		6000000 & 9.0737 \\ 
		6250000 & 9.0812 \\ 
		6500000 & 9.1306 \\ 
		6750000 & 9.1560 \\ 
		7000000 & 9.1771 \\ 
		7250000 & 9.1927 \\ 
		7500000 & 9.2173 \\ 
		7750000 & 9.2335 \\ 
		8000000 & 9.2549 \\ 
		8250000 & 9.2542 \\ 
		8500000 & 9.2936 \\ 
		8750000 & 9.3070 \\ 
		9000000 & 9.3145 \\ 
		9250000 & 9.3394 \\ 
		9500000 & 9.3354 \\ 
		9750000 & 9.4071 \\ 
		\hline
	\end{tabular}

    \caption{Tiempos de ejecución para el algoritmo \ref{alg:1a-dyv}, implementación \ref{cod:1a-dyv}.}
    \label{tab:1a-dyv}
\end{table}

\begin{figure}
	\centering
	\includegraphics[scale=0.76]{img/e1a-dyv}
	\caption{Gráfica de tiempos de ejecución para el caso Divide y Vencerás, 
		con f(x) y coeficiente de regresión especificada por \ref{eq:1a-dyv-regresion}.}
	\label{fig:1a-dyv-graph}
\end{figure}

Debido a que la eficiencia del algoritmo es logarítmica (tal y como hemos calculado en \ref{eq:1a-eficiencia-lineal}),
hemos podido calcular hasta $n = 10.000.000$. Para valores muy grandes se han empleado menor número de repeticiones
debido a que para valores muy grandes las fluctuaciones que se daban para casos pequeños son despreciables. 
Motiva también tomar esta medida la extrema ineficiencia del generador de vectores aleatorios. 

\newpage

\subsubsection{Comparativa}

El algoritmo que hace uso de la técnica Divide y Vencerás tiene un orden de eficiencia logarítmico, mientras
que la eficiencia del algoritmo básico es lineal, luego \textbf{se mejora notablemente la eficiencia usando la técnica de 
Divide y Vencerás} para solventar el problema. 
   
Este hecho se ilustra a partir de las gráficas correspondientes a los resultados empíricos obtenidos, veáse 
\ref{fig:1a-obv-graph} y \ref{fig:1a-dyv-graph}.

\subsection{Vector con valores repetidos}

Analizamos ahora el caso para vectores que pueden tomar valores repetidos. Veremos que, mientras que el algoritmo
\ref{alg:1a-obvio} sigue siendo válido, el algoritmo \ref{alg:1a-dyv} deja de serlo. 

\subsubsection{Caso obvio}

El algoritmo \ref{alg:1a-obvio} sigue preservando su validez, puesto que el hecho de que los elementos estén o no
repetidos no influye en la ejecución de este algoritmo. Tanto el pseudocódigo y el análisis de eficiencia es totalmente
análogo al apartado \ref{sec:1a-obvio}. 

\subsubsubsection{Pseudocódigo}

Es el mismo que el pseudocódigo \ref{alg:1a-obvio}. 

\subsubsubsection{Implementación}

La implementación sería el mismo que en \ref{cod:1a-obvio}. Otra implementación de este podría encontrarse en \ref{cod:1b-obvio}. 

\lstinputlisting[language=C++, firstline=82, lastline=90, 
caption=Implementación alternativa en C++ del algoritmo basado en Búsqueda Lineal., label={cod:1b-obvio}]
{../src/ejercicio-1-comp-fija-repetidos.cpp} 

\subsubsubsection{Análisis de Eficiencia}

El análisis de eficiencia \textbf{teórica} es análogo al realizado en \ref{sec:1a-obvio}, obteniéndose 
que el orden de eficiencia es lineal (\ref{eq:1a-obvio-eficiencia}). 

Para los análisis de eficiencia \textbf{empírico e híbrido}, se han obtenido las \textbf{funciones de ajuste}
y \textbf{coeficiente de regresión} especificadas por \ref{eq:1b-obvio-regresion}. Una representación
gráfica de los datos obtenidos, así como de la función de regresión, se puede encontrar en \ref{fig:1b-obvio-graph}. 

En términos experimentales, hemos realizado la experiencia hasta $n=10.000.000$, puesto que para tamaños superiores
el tiempo de ejecución del algoritmo sería demasiado elevado para realizar la experiencia. Por ejemplo, para un valor
$n=10^9$, se estima que el tiempo de ejecución sea \textbf{3 horas} (que difiere del tiempo de ejecución del algoritmo,
$T(10^9) = 5,8$ s, pues la penalización de la generación de vectores aleatorio es muy significativa). 
Para valores pequeños hemos realizado $n=1$ para comparar la cantidad de recursos que se emplea con un valor mónico. 

\begin{equation}
    \boxed{f(x) = 3,4926 \cdot 10 ^{-6} x + 0,9999, R^2 = 0,9999}
    \label{eq:1b-obvio-regresion}
\end{equation}

\begin{table}
	\footnotesize
	\centering
	\begin{tabular}{|r|r|}
        \hline
        $N$ & $T (ms)$ \\
        \hline
		1 & 0.0010 \\ 
		250000 & 0.8898 \\ 
		500000 & 1.8048 \\ 
		750000 & 2.7217 \\ 
		1000000 & 3.5974 \\ 
		1250000 & 4.4986 \\ 
		1500000 & 5.4154 \\ 
		1750000 & 6.3388 \\ 
		2000000 & 7.2286 \\ 
		2250000 & 8.1424 \\ 
		2500000 & 9.0827 \\ 
		2750000 & 9.9505 \\ 
		3000000 & 10.8597 \\ 
		3250000 & 11.8371 \\ 
		3500000 & 12.6701 \\ 
		3750000 & 13.6984 \\ 
		4000000 & 14.4887 \\ 
		4250000 & 15.4146 \\ 
		4500000 & 16.3083 \\ 
		4750000 & 17.2963 \\ 
		5000000 & 18.1089 \\ 
		5250000 & 19.1464 \\ 
		5500000 & 19.9771 \\ 
		5750000 & 20.8285 \\ 
		6000000 & 21.9765 \\ 
		6250000 & 22.8171 \\ 
		6500000 & 23.5401 \\ 
		6750000 & 24.6010 \\ 
		7000000 & 25.5379 \\ 
		7250000 & 26.5348 \\ 
		7500000 & 27.5240 \\ 
		7750000 & 28.0797 \\ 
		8000000 & 29.1724 \\ 
		8250000 & 29.8814 \\ 
		8500000 & 30.9434 \\ 
		8750000 & 31.8024 \\ 
		9000000 & 32.5969 \\ 
		9250000 & 33.6025 \\ 
		9500000 & 34.5628 \\ 
		9750000 & 36.4363 \\ 
        \hline
	\end{tabular}

    \caption{Tiempos de ejecución para el algoritmo \ref{alg:1a-obvio}, implementación \ref{cod:1b-obvio}.}
    \label{tab:1b-obvio}
\end{table}

\begin{figure}
	\centering
	\includegraphics[scale=0.76]{img/e1b-obvio}
	\caption{Gráfica de tiempos de ejecución para el caso obvio, 
		con f(x) y coeficiente de regresión especificada por \ref{eq:1b-obvio-regresion}.}
	\label{fig:1b-obvio-graph}
\end{figure}

\subsubsection{Caso Divide y Vencerás}

Para vectores ordenados con elementos repetidos, el algoritmo \ref{alg:1a-dyv} falla en algunas ocasiones. En este caso, 
el algoritmo \ref{alg:1a-dyv} 
\textbf{no siempre es válido}. Un ejemplo ilustrativo de ello se
puede observar en la figura \ref{fig:fallo-1a}, donde el algoritmo dictamina que no existe ningún
elemento que verifica la condición $v[i] = i$, cuando $v[0] = 0$. 

\begin{figure}
    \centering
    \includegraphics{img/esquema_fallo1a.pdf}
    \caption{Vector con elementos repetidos para el que falla 
    el algoritmo \ref{alg:1a-dyv}. Elaboración propia.}
    \label{fig:fallo-1a}
\end{figure}

Para solventar este problema, se ha de modificar el algoritmo \ref{alg:1a-dyv}, de manera que, cuando en una de las mitades
no encuentra un elemento que verifica la condición, también busque en la otra.

\subsubsubsection{Pseudocódigo}

\lstinputlisting[caption=Pseudocódigo asociado al caso Divide y Vencerás., label={alg:1b-dyv}]{listing/ejer1b-pseudo-dyv.txt}

\subsubsubsection{Implementación}

\lstinputlisting[language=C++, firstline=100, lastline=120, 
caption=Implementación en C++ del algoritmo basado en Divide y Vencerás., label={cod:1b-dyv}]
{../src/ejercicio-1-comp-fija-repetidos.cpp} 

\subsubsubsection{Determinación del umbral}

En este caso, a partir de los datos experimentales obtenidos en las tablas \ref{tab:1b-obvio} y \ref{tab:1b-dyv}, y cuya comparación
gráfica queda reflejado en \ref{fig:1b-comp}, el tiempo de ejecución del algoritmo empleado Divide y Vencerás
es siempre mayor o igual que el tiempo empleado por el algoritmo obvio. Las \textbf{razones} de esta situación están
explicadas en \ref{sec:1b-comp}. Por tanto, tenemos que el \textbf{umbral} queda especificado en \ref{eq:1b-umbral}. 

\begin{table}
					
	\centering
	\begin{tabular}{|r|r|r|}
        \hline
        $N$ & $T_1 (ms)$ & $T_2 (ms)$ \\
        \hline
		1 & 0.0010 & 0.0017 \\ 
		250000 & 0.8898 & 1.2638 \\ 
		500000 & 1.8048 & 2.5020 \\ 
		750000 & 2.7217 & 3.8801 \\ 
		1000000 & 3.5974 & 4.7835 \\ 
		1250000 & 4.4986 & 5.6027 \\ 
		1500000 & 5.4154 & 7.8714 \\ 
		1750000 & 6.3388 & 9.1403 \\ 
		2000000 & 7.2286 & 9.6127 \\ 
		2250000 & 8.1424 & 10.3977 \\ 
		2500000 & 9.0827 & 10.8430 \\ 
		2750000 & 9.9505 & 12.5853 \\ 
		3000000 & 10.8597 & 15.4864 \\ 
		3250000 & 11.8371 & 17.7592 \\ 
		3500000 & 12.6701 & 18.3230 \\ 
		3750000 & 13.6984 & 18.6021 \\ 
		4000000 & 14.4887 & 19.0520 \\ 
		4250000 & 15.4146 & 19.8521 \\ 
		4500000 & 16.3083 & 20.6940 \\ 
		4750000 & 17.2963 & 20.9706 \\ 
		5000000 & 18.1089 & 21.6910 \\ 
		5250000 & 19.1464 & 22.3115 \\ 
		5500000 & 19.9771 & 25.4420 \\ 
		5750000 & 20.8285 & 27.9453 \\ 
		6000000 & 21.9765 & 30.8848 \\ 
		6250000 & 22.8171 & 33.7447 \\ 
		6500000 & 23.5401 & 34.7666 \\ 
		6750000 & 24.6010 & 35.5849 \\ 
		7000000 & 25.5379 & 35.9877 \\ 
		7250000 & 26.5348 & 37.1252 \\ 
		7500000 & 27.5240 & 37.9441 \\ 
		7750000 & 28.0797 & 38.4063 \\ 
		8000000 & 29.1724 & 38.9629 \\ 
		8250000 & 29.8814 & 39.6185 \\ 
		8500000 & 30.9434 & 40.0429 \\ 
		8750000 & 31.8024 & 41.1611 \\ 
		9000000 & 32.5969 & 41.9953 \\ 
		9250000 & 33.6025 & 42.0843 \\ 
		9500000 & 34.5628 & 42.5158 \\ 
		9750000 & 36.4363 & 42.8710 \\
        \hline 
	\end{tabular}

    \caption{Comparación de tiempos de ejecución entre los algoritmos \ref{cod:1b-obvio} 
    (con tiempo asociado $T_1$) y \ref{cod:1b-dyv} (con tiempo de
    ejecución asociado $T_2$).}
    \label{tab:1b-comp}
\end{table}

Para la determinación analítica, consideramos las funciones de ajuste \ref{eq:1b-obvio-regresion} y \ref{eq:1b-dyv-eficiencia}. Por tanto,
queremos encontrar $(x,y) \in \mathbb R ^2$, con $x,y>0$ tal que se 
verifique el sistema de ecuaciones:

\begin{eqnarray}
	 y = 3,4926 \cdot 10 ^{-6} x + 0,9999 \\
	 y = 4,6553 \cdot 10^{-6} x + 0.8657
\end{eqnarray} 

\begin{equation}
    \boxed{UMBRAL \rightarrow +\infty}
    \label{eq:1b-umbral}
\end{equation}

Aplicando el método de Cramer, es fácil ver que la solución no tiene valores
positivos. Por tanto, por convenio, establecemos \ref{eq:1b-umbral}.

Las repercusiones de este resultado quedan nuevamente discutidas en \ref{sec:1b-comp}.

\begin{figure}
    \centering
    \includegraphics[scale=0.76]{img/e1b-comp.pdf}
    \caption{Gráfica de comparación tiempos de ejecución para Divide y Vencerás y caso obvio.}
    \label{fig:1b-comp}
\end{figure}

\subsubsubsection{Análisis de Eficiencia}

El análisis de eficiencia \textbf{teórico} es análogo al realizado en \ref{sec:1a-dyv}, \textbf{con la salvedad} de que, en el
peor caso, se ha de explorar también la otra mitad del vector, por lo que la expresión \ref{eq:1a-efi-dyv-rec}
se transformaría en \ref{eq:1b-efi-dyv-rec} (con $c \in \mathbb N$ una constante). 

\begin{equation}
    T(n) = \left\{ \begin{array}{lr} 2 T(n/2) + c & \text{si } n > \text{UMBRAL}\\ n & \text{si } n \leqslant \text{UMBRAL} \end{array} \right.
    \label{eq:1b-efi-dyv-rec}
\end{equation}

Para resolver \ref{eq:1b-efi-dyv-rec}, para $n > \text{UMBRAL}$
consideramos el cambio
de variable $n = 2^k$, por lo que la expresión queda $T(2^k) = 2T(2^{k-1}) + c$. Haciendo
un nuevo cambio de variable $t_m = T(2^m)$, tenemos que:
$t_m = 2 t_{m-1} + c$. Tenemos que la ecuación característica
asociada a la homogénea es $x-2 = 0$ y una solución particular 
es $\{c_o\}$, con $c_o \in \mathbb R$ constante. Por tanto, $t_m = T(2^m) = c_0 + c_1 2^m$,
con $c_0, c_1 \in \mathbb R$, de donde deducimos que 

\begin{equation}
    T(n) = c_0 + c_1 n \Rightarrow \boxed{T(n) \in O(n)}
\end{equation}

El análisis de eficiencia \textbf{empírico e híbrido}
de la implementación \ref{cod:1b-obvio} puede encontrarse en la tabla \ref{tab:1b-dyv}. La 
\textbf{función de ajuste} y el \textbf{coeficiente de regresión} vienen dados 
por \ref{eq:1b-dyv-eficiencia}. Una representación gráfica de estos datos se ha realizado en
\ref{fig:1b-dyv-graph}. 

\begin{equation}
    \boxed{f(x) = 4,6553 \cdot 10^{-6} x + 0.8657, R^2 = 0,9834}
    \label{eq:1b-dyv-eficiencia}
\end{equation}

Para tomar las medidas experimentales, como el orden de eficiencia calculado 
(\ref{eq:1b-dyv-eficiencia}) es lineal, hemos tomado las medidas hasta 
un valor razonable $n=10.000.000$. En este caso, las limitaciones de tiempo
no han sido tan pronunciadas como en el caso de vectores no repetidos,
puesto que el generador de números aleatorios es más eficiente que el caso
anterior, por lo que hemos podido llevar hasta un valor bastante elevado 
incluso para un orden de eficiencia peor. 

\begin{table}
	\footnotesize
	\centering
	\begin{tabular}{|r|r|}
        \hline
        $N$ & $T(ms)$ \\
        \hline
		1 & 0.0017 \\ 
		250000 & 1.2638 \\ 
		500000 & 2.5020 \\ 
		750000 & 3.8801 \\ 
		1000000 & 4.7835 \\ 
		1250000 & 5.6027 \\ 
		1500000 & 7.8714 \\ 
		1750000 & 9.1403 \\ 
		2000000 & 9.6127 \\ 
		2250000 & 10.3977 \\ 
		2500000 & 10.8430 \\ 
		2750000 & 12.5853 \\ 
		3000000 & 15.4864 \\ 
		3250000 & 17.7592 \\ 
		3500000 & 18.3230 \\ 
		3750000 & 18.6021 \\ 
		4000000 & 19.0520 \\ 
		4250000 & 19.8521 \\ 
		4500000 & 20.6940 \\ 
		4750000 & 20.9706 \\ 
		5000000 & 21.6910 \\ 
		5250000 & 22.3115 \\ 
		5500000 & 25.4420 \\ 
		5750000 & 27.9453 \\ 
		6000000 & 30.8848 \\ 
		6250000 & 33.7447 \\ 
		6500000 & 34.7666 \\ 
		6750000 & 35.5849 \\ 
		7000000 & 35.9877 \\ 
		7250000 & 37.1252 \\ 
		7500000 & 37.9441 \\ 
		7750000 & 38.4063 \\ 
		8000000 & 38.9629 \\ 
		8250000 & 39.6185 \\ 
		8500000 & 40.0429 \\ 
		8750000 & 41.1611 \\ 
		9000000 & 41.9953 \\ 
		9250000 & 42.0843 \\ 
		9500000 & 42.5158 \\ 
		9750000 & 42.8710 \\ 
        \hline
	\end{tabular}
    \caption{Tiempos de ejecución para el algoritmo \ref{alg:1b-dyv}.}
    \label{tab:1b-dyv}
\end{table}

\begin{figure}
    \centering
    \includegraphics[scale=0.76]{img/e1b-dyv}
    \caption{Gráfica de tiempos de ejecución para el caso Divide y Vencerás, 
    con f(x) y coeficiente de regresión especificada por \ref{eq:1b-dyv-eficiencia}.}
    \label{fig:1b-dyv-graph}
\end{figure}

\subsubsection{Comparativa} \label{sec:1b-comp}

Al contrario que en el caso de vectores no repetidos, visualizando la tabla
con comparación \ref{tab:1b-comp} y está representado en la gráfica \ref{fig:1b-comp}, 
los \textbf{tiempos 
de ejecución son peores} en el caso Divide y
Vencerás que en el caso obvio. El origen de esto se debe a las siguientes
razones:

\begin{itemize}
    \item \textbf{Mismo orden de eficiencia}. Al contrario que en el caso con valores 
    no repetidos, el orden de eficiencia
    es la misma tanto en el caso obvio como en el caso trivial, tal y como se puede
    observar en la expresión \ref{eq:2a-eficiencia}, por lo que este resultado es plausible. 
    \item \textbf{Sobrecarga de la pila}. En los vectores generados, la distribución 
    general es la aparición del índice
    buscado en los elementos finales del vector, lo que obliga tanto al caso obvio como al caso
    Divide y Vencerás a analizar casi todo el vector, presentando éste último el 
    inconveniente de, en cada iteración, \textbf{introducir información de contexto
    en la pila} por cada llamada recursiva, lo que penaliza el rendimiento global del
    algoritmo. 
\end{itemize}

Como consecuencia de esta situación, tenemos que el algoritmo \ref{alg:1a-obvio} es siempre mejor o igual 
que el algoritmo \ref{alg:1b-dyv}, por lo que la técnica Divide y Vencerás \textbf{no logra superar en 
rendimiento} al caso obvio en ningún caso. Por tanto, en cualquier situación será más recomendable
implementar el caso obvio en lugar de emplear esta técnica. 
