\usepackage{listings}

\subsection{Apartado A}

\textb{Problema}: Dado un vector ordenado (de forma no decreciente) de $n$ números enteros $v$, todos 
distintos, el objetivo es determinar si existe un índice i tal que $v[i] = i$ y 
encontrarlo en ese caso. 

Para resolver el problema hemos utilizado dos algoritmos, uno básico basado en 
la búsqueda lineal, y otro basado en la técnica de Divide y Vencerás. 

El último algoritmo se ha desarrollado a partir de un análisis detallado del 
problema, a partir del cual hemos conseguido demostrar las siguientes proposiciones:

\textbf{Proposición 1.1.} Sea i con $0 \leqslant i < n$ fijo, tal que $v[i]=i+k$ con 
$k \in \mathbb N$, entonces $v[j]!=j$, $\forall j$ con $i < j < n$. 

\textit{Demostración}: Razonemos por contradicción. Supongamos que para $j > i$ se 
cumple que $v[j]=j$, entonces \exists $m \in \mathbb N$ tal que $j=i+m$. Como el 
vector está ordenado en orden no decreciente sin repetidos, como mínimo un elemento
difiere en una unidad del elemento siguiente. Podemos considerar esto último sin 
lugar a ambiguedad para todo elemento del vector, luego $v[j]=v[i]+m$. Como por hipótesis
$v[i]=i+k$, tendríamos que $v[j]=i+k+m$, pero $i=j-m$, entonces $v[j]=j-m+k+m=j+k$, lo
cual es una clara contradicción, ya que habiamos supuesto que $v[j]=j$. Por tanto, no 
existe ningún $j$ tal que $v[j]=j$.

\textbf{Proposición 1.2.} Sea i con $0 < i < n$ fijo, tal que $v[i]=i-k$ con 
$k \in \mathbb N$, entonces $v[j]!=j$,  $\forall j$ con $0 \leqslant j < i$. 

\textit{Demostración}: Razonemos por contradicción. Supongamos que para $j < i$ se 
cumple que $v[j]=j$, entonces \exists $m \in \mathbb N$ tal que $j=i-m$. Como el 
vector está ordenado en orden no decreciente sin repetidos, como mínimo un elemento
difiere en una unidad del elemento siguiente. Podemos considerar esto último sin 
lugar a ambiguedad para todo elemento del vector, luego $v[j]=v[i]-m$. Como por hipótesis
$v[i]=i-k$, tendríamos que $v[j]=i-k-m$, pero $i=j+m$, entonces $v[j]=j+m-k-m=j-k$, lo
cual es una clara contradicción, ya que habiamos supuesto que $v[j]=j$. Por tanto, no 
existe ningún $j$ tal que $v[j]=j$.

A continuación, mostraremos los dos algoritmos usados, en psudocódigo e implementados en 
lenguaje c++: 

\subsubsection{Algoritmo basado en la búsqueda lineal}

\begin{lstlisting}
    BUSQUEDA_LINEAL(v[1...n-1], ini, fin)
        indice = ini
        si el tamaño del vector es mayor que 0
            para i = ini hasta fin o hasta que encontremos la posición
                si v[i] es igual a i (encontramos la posición)
                    indice = i

        devolver indice
\end{lstlisting}

%\lstinputlisting[language=C++, firstline=88, lastline=181, caption=Implementación en C++ del algoritmo basado en la búsqueda lineal.]{../src/<nombre_programa>.cpp} 

\subsubsection{Algoritmo basado en la técnica Divide y Vencerás}

\begin{lstlisting}
    
\end{lstlisting}



