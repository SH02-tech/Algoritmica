El \textbf{problema} que se nos plantea es, dado un vector ordenado (de forma no decreciente) de $n$ números enteros $v$, todos 
distintos, determinar si existe un índice i tal que $v[i] = i$ y 
encontrarlo en ese caso. Para resolver el problema hemos utilizado dos algoritmos, uno básico basado en 
la búsqueda lineal, y otro basado en la técnica de Divide y Vencerás. 

\subsection{Vector con valores no repetidos}

En primer lugar, abordamos el problema para vectores con valores no repetidos. Para esta parte
caben destacar los lemas \ref{lem:1} y \ref{lem:2}. 

% El último algoritmo se ha desarrollado a partir de un análisis detallado del 
% problema, a partir del cual hemos conseguido demostrar las siguientes proposiciones:

\begin{lemma}
    \label{lem:1}
    Sea i con $0 \leqslant i < n$ fijo, tal que $v[i]=i+k$, con $k \in \mathbb N$. 
    Entonces $v[j] \neq j$, $\forall j \in \mathbb N$ con $i < j < n$. 
\end{lemma}

\begin{proof}
    Razonemos por contradicción. Supongamos que para $j > i$ se 
    cumple que $v[j]=j$, entonces $\exists m \in \mathbb N$ tal que $j=i+m$. Como el 
    vector está ordenado en orden no decreciente sin repetidos, como mínimo un elemento
    difiere en una unidad del elemento siguiente. Podemos considerar esto último sin 
    lugar a ambiguedad para todo elemento del vector, luego $v[j]=v[i]+m$. Como por hipótesis
    $v[i]=i+k$, tendríamos que $v[j]=i+k+m$, pero $i=j-m$, entonces $v[j]=j-m+k+m=j+k$, lo
    cual es una clara contradicción, ya que habiamos supuesto que $v[j]=j$. Por tanto, no 
    existe ningún $j$ tal que $v[j]=j$.
\end{proof}

\begin{lemma}
    \label{lem:2}
    Sea i con $0 < i < n$ fijo, tal que $v[i]=i-k$ con 
    $k \in \mathbb N$, entonces $v[j] \neq j$,  $\forall j$ con $0 \leqslant j < i$. 
\end{lemma}

\begin{proof}
    Razonemos por contradicción. Supongamos que para $j < i$ se 
    cumple que $v[j]=j$, entonces $\exists m \in \mathbb N$ tal que $j=i-m$. Como el 
    vector está ordenado en orden no decreciente sin repetidos, como mínimo un elemento
    difiere en una unidad del elemento siguiente. Podemos considerar esto último sin 
    lugar a ambiguedad para todo elemento del vector, luego $v[j]=v[i]-m$. Como por hipótesis
    $v[i]=i-k$, tendríamos que $v[j]=i-k-m$, pero $i=j+m$, entonces $v[j]=j+m-k-m=j-k$, lo
    cual es una clara contradicción, ya que habiamos supuesto que $v[j]=j$. Por tanto, no 
    existe ningún $j$ tal que $v[j]=j$.
\end{proof}

\subsection{Caso obvio}

En este caso el algoritmo de búsqueda del índice se basa en la búsqueda lineal. 
\subsubsection{Pseudocódigo}

\begin{lstlisting}
BUSQUEDA_LINEAL(v[1...n-1], ini, fin)
    indice = ini
    si el tamaño del vector es mayor que 0
        para i = ini hasta fin o hasta que encontremos la posición
            si v[i] es igual a i (encontramos la posición)
                indice = i
    devolver indice
\end{lstlisting}

\subsubsection{Implementación}
\subsubsection{Análisis de Eficiencia}

%\lstinputlisting[language=C++, firstline=88, lastline=181, caption=Implementación en C++ del algoritmo basado en la búsqueda lineal.]{../src/<nombre_programa>.cpp} 

\subsection{Caso Divide y Vencerás}

En esta parte, como consecuencia de los lemas \ref{lem:1} y \ref{lem:2}, el algoritmo presentado es similar 
al algoritmo de búsqueda binaria (un caso muy particular de Divide y Vencerás, 
llamado \textbf{simplificación} \cite{Verdegay2017}).

\subsubsection{Pseudocódigo}
\subsubsection{Implementación}
\subsubsection{Análisis de Eficiencia}
En este caso, para tomar las medidas experimentales, se ha tenido que tomar para cada $n$ una \textbf{media de 1.000.000
de valores}. Esto se debe a que el nº de operaciones del algoritmo fluctúa mucho y puede terminar mucho antes o después en función
de los valores de entrada. \textbf{Para determinar este valor}, hemos probado con diferentes número de entradas, 
viendo que para este valor las salidas del tiempo de ejecución son más o menos \textbf{estables}. 

\begin{lstlisting}
    
\end{lstlisting}