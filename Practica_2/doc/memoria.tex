\documentclass{homework}

\title{Práctica 2: Divide y Vencerás}
\author{Shao Jie Hu Chen \\ Mario Megías Mateo \\ Jesús Samuel García Carballo}
\renewcommand{\course}{Algorítmica}
% \date{1 de abril de 2022}

\begin{document}
    \maketitle
    \tableofcontents
    \newpage

    \section{Introducción}
    En ciencias de la computación, la técnica \textbf{Divide y Vencerás} es un paradigma de diseño
algorítmico que consiste en (i) dividir un problema en pequeñas partes que sean más manejables,
(ii) resolver cada subproblema individualmente y (iii) unificar las soluciones obtenidas 
para obtener el resultado final del algoritmo original \cite{Cormen2017}. 

El \textbf{objetivo} de esta práctica es resolver una serie de problemas algorítmicos aplicando
Divide y Vencerás, comparando la resolución que podemos considerar obvia respecto a la obtenida 
aplicando esta técnica. 

\subsection{Metodología}

Para realizar esta práctica se han implementado las soluciones a cada uno de los problemas
propuestos y se ha analizado su eficiencia respecto a los algoritmos ''obvios'' para resolverlos.
Con la finalidad de \textbf{automatizar} la ejecución y la generación de datos de eficiencia
de interés, hemos empleado el mismo programa que desarrollamos en \cite{Rojo2022}.
El funcionamiento interno del \textbf{Analizador} queda ilustrado en la figura 
\ref{fig:analizador} (más información en la referencia \cite{Rojo2022}). 

\begin{figure}[h]
    \centering
    \includegraphics[scale=0.87]{img/esquema_graphkiller.pdf}
    \caption{Esquema de funcionamiento del \textbf{Analizador} \cite{Rojo2022}. Elaboración propia.}
    \label{fig:analizador}
\end{figure}

\subsection{Equipo empleado}

Para el desarrollo de esta práctica, se ha empleado un equipo de la marca HP (modelo Pavilion Gaming Laptop 15-dk0xxx) 
dotado de las siguientes prestaciones: 

\begin{itemize}
    \item \textbf{Procesador}: Intel(R) Core(TM) i7-9750H CPU @ 2.60GHz
    \item \textbf{Memoria RAM:} 32 GB DDR4
    \item \textbf{Sistema Operativo}: Ubuntu 20.04.4 LTS
\end{itemize}
    \newpage

    \section{Metodología}
    \input{cap/2-metodologia.tex}
    \newpage

    \section{Ejercicio 1: Vector de punto fijo}
    \input{cap/3-ejercicio1.tex}
    \newpage
    
    \section{Ejercicio 2: Mezcla ordenada de vectores}
    \input{cap/4-ejercicio2.tex}
    \newpage

    \section{Conclusión}
    \input{cap/5-conclusion.tex}
    \newpage

    \newpage
    \begin{thebibliography}{0}
        \bibitem{Verdegay2017} Verdegay Galdeano. (2017). Lecciones de Algorítmica / José Luis Verdegay. Técnica Avicam.
        \bibitem{Cormen2017} Cormen. (2017). Introduction to algorithms / Thomas H. Cormen... [et al.] (3rd ed.). PHI Learning.
        \bibitem{Garrido2018} Garrido Carrillo. (2018). Estructuras de datos avanzadas: con soluciones en C++ / A. Garrido. Universidad de Granada.        
    \end{thebibliography}
\end{document}