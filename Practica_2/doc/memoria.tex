\documentclass{homework}

\title{Práctica 2: Divide y Vencerás}
\author{Shao Jie Hu Chen \\ Mario Megías Mateo \\ Jesús Samuel García Carballo}
\renewcommand{\course}{Algorítmica}
% \date{1 de abril de 2022}

\begin{document}
    \maketitle
    \tableofcontents
    \newpage

    \section{Introducción}
    En ciencias de la computación, la técnica \textbf{Divide y Vencerás} es un paradigma de diseño
algorítmico que consiste en (i) dividir un problema en pequeñas partes que sean más manejables,
(ii) resolver cada subproblema individualmente y (iii) unificar las soluciones obtenidas 
para obtener el resultado final del algoritmo original \cite{Cormen2017}. 

El \textbf{objetivo} de esta práctica es resolver una serie de problemas algorítmicos aplicando
Divide y Vencerás, comparando la resolución que podemos considerar obvia respecto a la obtenida 
aplicando esta técnica. 

\subsection{Metodología}

Para realizar esta práctica se han implementado las soluciones a cada uno de los problemas
propuestos y se ha analizado su eficiencia respecto a los algoritmos ''obvios'' para resolverlos.
Con la finalidad de \textbf{automatizar} la ejecución y la generación de datos de eficiencia
de interés, hemos empleado el mismo programa que desarrollamos en \cite{Rojo2022}.
El funcionamiento interno del \textbf{Analizador} queda ilustrado en la figura 
\ref{fig:analizador} (más información en la referencia \cite{Rojo2022}). 

\begin{figure}[h]
    \centering
    \includegraphics[scale=0.87]{img/esquema_graphkiller.pdf}
    \caption{Esquema de funcionamiento del \textbf{Analizador} \cite{Rojo2022}. Elaboración propia.}
    \label{fig:analizador}
\end{figure}

\subsection{Equipo empleado}

Para el desarrollo de esta práctica, se ha empleado un equipo de la marca HP (modelo Pavilion Gaming Laptop 15-dk0xxx) 
dotado de las siguientes prestaciones: 

\begin{itemize}
    \item \textbf{Procesador}: Intel(R) Core(TM) i7-9750H CPU @ 2.60GHz
    \item \textbf{Memoria RAM:} 32 GB DDR4
    \item \textbf{Sistema Operativo}: Ubuntu 20.04.4 LTS
\end{itemize}
    \newpage

    \section{Ejercicio 1: Vector de punto fijo}
    \subsection{Apartado A}

\textb{Problema}: Dado un vector ordenado (de forma no decreciente) de $n$ números enteros $v$, todos 
distintos, el objetivo es determinar si existe un índice i tal que $v[i] = i$ y 
encontrarlo en ese caso. 

Para resolver el problema hemos utilizado dos algoritmos, uno básico basado en 
la búsqueda lineal, y otro basado en la técnica de Divide y Vencerás. 

El último algoritmo se ha desarrollado a partir de un análisis detallado del 
problema, a partir del cual hemos conseguido demostrar las siguientes proposiciones:

\textbf{Proposición 1.1.} Sea i con $0 \leqslant i < n$ fijo, tal que $v[i]=i+k$ con 
$k \in \mathbb N$, entonces $v[j]!=j$, $\forall j$ con $i < j < n$. 

\textit{Demostración}: Razonemos por contradicción. Supongamos que para $j > i$ se 
cumple que $v[j]=j$, entonces \exists $m \in \mathbb N$ tal que $j=i+m$. Como el 
vector está ordenado en orden no decreciente sin repetidos, como mínimo un elemento
difiere en una unidad del elemento siguiente. Podemos considerar esto último sin 
lugar a ambiguedad para todo elemento del vector, luego $v[j]=v[i]+m$. Como por hipótesis
$v[i]=i+k$, tendríamos que $v[j]=i+k+m$, pero $i=j-m$, entonces $v[j]=j-m+k+m=j+k$, lo
cual es una clara contradicción, ya que habiamos supuesto que $v[j]=j$. Por tanto, no 
existe ningún $j$ tal que $v[j]=j$.






    \newpage
    
    \section{Ejercicio 2: Mezcla ordenada de vectores}
    En esta sección abordaremos el problema del viajante de comercio. Dado que
se trata de un algoritmo de clase NP, es prácticamente inviable encontrar 
un algoritmo que encuentre la solución óptima en un tiempo razonable.

Por tanto, para abordar este problema se emplearán 3 heurísticas diferentes que,
si bien no siempre encuentran la solución óptima, determinan soluciones 
aceptables en tiempos razonables. Para cada una de
estas heurísticas se comparará su rendimiento en un conjunto de prueba,
midiendo los tiempos de ejecución. 

\subsection{Descripción de las heurísticas}

En esta sección se realizará una descripción de cada heurística que hemos 
empleado, incluyendo tanto algoritmos, implementación como casos de ejemplo.
Dos heurísticas han sido proporcionados por los profesores: el \textit{vecino
más cercano} e \textit{inserción}, mientras que la tercera es de
elaboración propia.

\subsubsection{Vecino más cercano}

\begin{figure}[H] 
  \centering
  \includegraphics[scale=1.5]{img/DibVecCercano.pdf}
  \caption{Ilustración del funcionamiento de la heurística del vecino más cercano.}
  \label{fig:vec}
\end{figure}

La filosofía que gobierna esta heurística es bastante sencilla. La idea 
radica en ir insertando ciudades en el conjunto de solución $S$, de manera
que para decidir la ciudad $c_{i+1}$ que se insertará se escoge aquella que
dista la menor distancia de la ciudad $c_i$ (para $c_0$ se escoge
un nodo al azar). Esta idea queda resumida en el Algoritmo~\ref{alg:vec_cercano}. 

\begin{algorithm}
    \caption{An algorithm with caption}\label{alg:vec_cercano}
    \KwData{$ L (lista de arcos ordenados)$}
    \KwResult{$y = x^n$}
    $y \gets 1$\;
    $X \gets x$\;
    $N \gets n$\;
    \While{$N \neq 0$}{
      \eIf{$N$ is even}{
        $X \gets X \times X$\;
        $N \gets \frac{N}{2}$ \Comment*[r]{This is a comment}
      }{\If{$N$ is odd}{
          $y \gets y \times X$\;
          $N \gets N - 1$\;
        }
      }
    }
\end{algorithm}

\subsubsubsection{Análisis de eficiencia}

Aquí va la teórica.


\subsubsection{Inserción económica}

\begin{figure}[H] 
  \centering
  \includegraphics[scale=1.5]{img/DibInsercion.pdf}
  \caption{Ilustración del funcionamiento de la heurística de inserción.}
  \label{fig:inser}
\end{figure}

En este caso para obtener la solución final se fijan $c_1,c_2,c_3$ ciudades
(siguiendo las indicaciones de la práctica, se han
escogido las ciudades situadas más al norte, este y oeste). 
Posteriormente,
para decidir la ciudad $c_{i+1}$ que se insertará en el conjunto de solución,
se elige aquella que, al insertarse en un determinado arco del ciclo 
$0 \leq j \leq i$, minimice la variación en el ciclo formado por las ciudades
$c_1,c_2,\cdots,c_{i}$. Dicha ciudad se añade intercalado entre las ciudades
$c_{j},c_{j+1}$. El algoritmo asociado a esta heurística se encuentra en 
Algoritmo~\ref{alg:insercion}. 

\begin{algorithm}[H]
  \caption{Algoritmo auxiliar para la inserción. InsercionEconomica}\label{alg:insercion-aux-1}
  \begin{minipage}{0.92\textwidth}
    \textbf{Parámetro}: ady (matriz de adyacencia)

    \textbf{Parámetro}: ciclo (solución parcial)

    \textbf{Parámetro}: pos
  \end{minipage}

  $(c_{ini}, c_{fin}) = (-1,-1)$\;
  $var = \infty$\;
  ind = -1\;

  \For{i desde 0 hasta ciclo.size-1}{
    $(c_a, c_b)$ = ciclo[i]\;
    nueva\_variacion = $ady[c_a][c_i] + ady[c_i][c_b] - ady[c_a][c_b]$\;

    \If{nueva\_variacion $<$ variacion}{
      $(c_{ini}, c_{fin}) = (c_a, c_b)$\;
      ind = i\;
    }
  }
  
  \Return{$(ind, pos, var)$}
\end{algorithm}

\begin{algorithm}[H]
  \caption{Algoritmo auxiliar para la inserción. CiudadEconomica (se usa el Algoritmo~\ref{alg:insercion-aux-1})}\label{alg:insercion-aux-2}
  \begin{minipage}{0.92\textwidth}
    \textbf{Parámetro}: ady (matriz de adyacencia)

    \textbf{Parámetro}: ciclo (solución parcial)

    \textbf{Parámetro}: banderas
  \end{minipage}

  ind = 0\;

  \While{banderas[ind] e ind valido}{
    ind++\;
  }

  $(ind_a, pos_a, var_a)$ = InsercionEconomica(ady,ciclo,ind)\;

  \For{i desde ind+1 hasta ady.tamaño}{
    \If{banderas[i] no activado}{
      $(ind_n, pos_n, var_n)$ = InsercionEconomica(ady, ciclo, i)\;
      % $(c_{n}, c_{n+1}, var_n)$ = InsercionEconomica(ady, ciclo, i)\;

      \If{$var_n < var_a$ }{
        $(ind_a, pos_a, var_a)$ = $(ind_n, pos_n, var_n$\;
      }
    }
  }

  \Return{$(c_{ind_a},pos_a)$}
  
\end{algorithm}

El Algoritmo~\ref{alg:insercion-aux-1} y el Algoritmo~\ref{alg:insercion-aux-2}
se emplean como funciones auxiliales sobre los que se basa el algoritmo maestro. 

Se presenta el algoritmo máster.

\begin{algorithm}[H]
    \caption{Algoritmo basado en inserción. CicloInsercionEconomica (se usan los algoritmos 
    auxiliares Algoritmo~\ref{alg:insercion-aux-1} y Algoritmo~\ref{alg:insercion-aux-2})}\label{alg:insercion}
    \begin{minipage}{0.92\textwidth}
      \textbf{Parámetro}: puntos (vector de posiciones)
      
      \textbf{Parámetro}: ady (matriz de adyacencia)
    \end{minipage}
    % $Ady = MatrizAdyacencia(C)$\;
    % $N = C.\text{tamaño}$\;
    % $S = []$\;
    % $S$.añadir($c_\alpha$), con $\alpha \in \{j \in \mathbb N : c_j.x \geq c.x, \forall c \in C\}$ \Comment*[r]{Ciudad más al este}
    % $S$.añadir($c_\beta$), con $\beta \in \{j \in \mathbb N : c_j.x \leq c.x, \forall c \in C\}$ \Comment*[r]{Ciudad más al oeste}
    % $S$.añadir($c_\gamma$), con $\gamma \in \{j \in \mathbb N : c_j.y \geq c.y, \forall c \in C\}$ \Comment*[r]{Ciudad más al norte}
    % \While{$S.\text{tamaño} \leq N$}{
    %   $(i,pos) = CiudadEconomica(C, S, Ady)$\;
    %   $S$.insertar(pos, $c_i$)\;
    %   $C$.quitar($c_i$)\;
    % }

    ciclo = []\;

    N = puntos.tamaño;
    Banderas = []\;

    \For(){i desde 0 hasta N-1}{
      Banderas.insertar(false)
    }

    \Comment*{Añadimos las tres primeras ciudades}

    ind\_este = IndCiudadEste(puntos);
    banderas[ind\_este] = true\;

    ind\_oeste = IndCiudadOeste(puntos);
    banderas[ind\_oeste] = true\;

    ind\_este = IndCiudadNorte(puntos);
    banderas[ind\_norte] = true\;


    ciclo.insertar((ind\_este, ind\_oeste));
    ciclo.insertar((ind\_oeste, ind\_norte))\;
    ciclo.insertar((ind\_norte, ind\_este))\;

    % $S$.añadir(puntos[ind_este])\;
    % $S$.añadir(puntos[ind_oeste])\;
    % $S$.añadir(puntos[ind_norte])\;

    \Comment*{Añadimos el resto de ciudades}
    \For(){i desde 3 hasta N-1}{
      $(c_i,pos)$ = CiudadEconomica(ady, ciclo, banderas)\;
      $(c_j, c_{j+1})$ = s[pos]\;
      banderas[$c_i$] = true\;
      ciclo.insertar(pos, $(c_i,c_{j+1})$)\;
      ciclo.insetar(pos, $(c_j,c_i)$)\;
    }
    \Return{ciclo}
\end{algorithm}

El código asociado al algoritmo Algoritmo~\ref{alg:insercion} viene especificado
a continuación, en el Código~\ref{cod:tsp1}. 

\lstinputlisting[label={cod:tsp1}, firstline=119, lastline=300, language=C++,
caption=Implementación del algoritmo de inserción en TSP.]{../src/tsp.cpp}

\subsubsection{Heurística basada en el algoritmo de Kruskal} 

\begin{figure}[H] 
  \centering
  \includegraphics[scale=1.5]{img/DibPropio.pdf}
  \caption{Ilustración del funcionamiento de la heurística propia, basada en el algoritmo de Kruskal.}
  \label{fig:vec}
\end{figure}

La siguiente heurística de elaboración propia se basa en la idea utilizada en el \textbf{algoritmo de Kruskal} para la determinación del AGM. 

Dado el conjunto de $n$ ciudades conectadas entre sí, se crea un grafo $G$ que represente la red de ciudades, y un grafo $S$ que en el momento de su construcción contenga únicamente las ciudades o nodos de $G$, pero sin ninguna conexión entre nodos.

A continuación, ordenamos los arcos de $G$ en orden creciente según la ponderación del arco (sin incluir lazos, es decir, ciclos entre un nodo y el mismo). Esta \textbf{lista de arcos}, $L$, constituye nuestro \textbf{conjunto de candidatos}, es decir en cada etapa del algoritmo, \textbf{la función de selección escoge el arco de menor ponderación} (primer elemento de la lista), y seguidamente comprobamos si el arco escogido cumple las condiciones de factibilidad, si las cumple será insertado en $S$, y en caso contrario ese arco será descartado de la solución. 

En cuanto a la factibilidad de un candidato, tenemos \textbf{dos condiciones de factibilidad}. La primera consiste en comprobar que los nodos entre los que vamos a insertar el arco cumplen la siguiente condición: el nodo origen debe de tener como máximo un nodo incidente, y no puede tener un arco adyacente, y el nodo destino debe de tener como máximo un arco adyacente y no debe tener un arco incidente. Con esta condición nos garantizamos parte de la construcción de un circuito en el grafo $S$, de forma que \textbf{cada nodo del grafo solución tenga una única entrada y una única salida}. La segunda condición consiste en comprobar que al añadir el arco candidato al grafo solución $S$ \textbf{no se formen ciclos}, aunque esta condición no se comprueba en la última iteración del algoritmo, ya que debe formarse un circuito en la última etapa.

El algoritmo terminará cuando se hayan insertado tantas aristas como nodos tiene $S$, formando un circuito Hamiltoniano. La elección del arco de menor coste entre los que hay disponibles responde a la idea de intentar minimizar la \textbf{función objetivo}, que mide el coste total del circuito.

\begin{algorithm}
	\caption{Algoritmo basado en Kruskal}\label{alg:bas_kruscal}
	\begin{minipage}{0.92\textwidth}
		\textbf{Parámetro}: L (lista de arcos ordenada)
		
		\textbf{Parámetro}: nNodos (número de nodos/ciudades)
		
		\textbf{Parámetro}: grafoSol (grafo con la solución, parámetro de entrada/salida)
	\end{minipage}
	nArcosInsertados = 0\;
	\While{narcosInsertados $<$ nNodos \&\& !L.vacia()}{
		arcoCandidato = L.frente()\;
		L.eliminarFrente()\;
		
		\eIf{$N$ is even}{
			$X \gets X \times X$\;
			$N \gets \frac{N}{2}$ \Comment*[r]{This is a comment}
		}{\If{$N$ is odd}{
				$y \gets y \times X$\;
				$N \gets N - 1$\;
			}
		}
	
	}
\end{algorithm}

\lstinputlisting[label={cod:cod-e2c}, firstline=7, lastline=145, language=C++,
caption=Implementación del algoritmo de TSP basado en Kruskal.]{../src/ejercicio-2-c.cpp}

\subsubsubsection{Análisis de eficiencia}

Empecemos con el análisis teórico, analizando la eficiencia asintótica $big(O)$. Llamaremos $n$ al número 
de nodos del grafo que representa la red. La función que se encargar de obtener el circuito hamiltoniano 
solución \texttt{resuelvePVC}, tiene un bucle \texttt{while} que realizará $n$ iteraciones. Dentro del mismo, 
la función \texttt{buscarNodoConIndice} tiene eficiencia $O(n)$, y también en el interior del bucle, la 
función \texttt{sinCiclos} tiene una eficiencia $O(n^2)$, luego \textbf{la eficiencia de \texttt{resuelvePVC} es $O(n^3)$}.

Estudiando los tiempos de cada algoritmo nos damos cuenta de que el algoritmo proporcionado por nosotros es el más ineficiente
con bastante diferencia respecto a los otros dos, siendo el del vecino el algoritmo que menos tarda en ejecutarse.

\begin{table}[H]
  \centering
  \begin{tabular}{|c|c|c|c|}
    \hline
    & bayg & eil & ulysses \\
    \hline
    vecino más cercano & 76 & 248 & 39 \\
    \hline
    inserción & 402 & 2240 & 120 \\
    \hline
    $\alpha$-Kruskal & 8859 & 16048 & 2559 \\
    \hline
  \end{tabular}
  \caption{Tabla donde se indica el tiempo empleado en microsegundos en los conjuntos de prueba en función de la heurística empleada.}
\end{table}


\begin{figure}[H]
  \centering
  \includegraphics[scale=0.15]{../src/Comparación_de_algoritmos.pdf}
  \caption{Tabla donde se indica el tiempo empleado en microsegundos en los conjuntos de prueba en función de la heurística empleada.}
\end{figure}

\subsection{Comparación de rendimientos}

Al tratarse de un problema de clase $NP$, no existe un algoritmo óptimo que encuentra siempre
la mejor solución en un tiempo razonable. Mediante las tres heurísticas hemos realizado una aproximación
para encontrar soluciones que, si bien no son siempre óptimas, ofrecen un buen rendimiento relativo.

En la siguiente tabla quedan recogidas las distancias calculadas por cada una de las heurísticas.

\begin{table}[H]
    \centering
    \begin{tabular}{|c|c|c|c|}
      \hline
      & bayg & eil & ulysses \\
      \hline
      vecino más cercano & 10200 & 662 & 79 \\
      \hline
      inserción & 9607 & 575 & 70 \\
      \hline
      $\alpha$-Kruskal & 10845 & 710 & 78 \\
      \hline
    \end{tabular}
    \caption{Tabla donde se indica la distancia total del ciclo recorrido en los conjuntos de prueba
    en función de la heurística empleada.}
\end{table}



También se han adjuntado capturas con cada una de las ejecuciones realizadas, mostrándose tanto la suma
total como el camino recorrido.

\begin{figure}[H]
  \centering
  \includegraphics[scale=0.5]{img/dist-vecinos-bayg.png}
  \caption{Camino generado por la heurística del vecino más cercano en bayg.}
\end{figure}

\begin{figure}[H] 
  \centering
  \includegraphics[scale=0.5]{img/dist-vecinos-eil.png}
  \caption{Camino generado por la heurística del vecino más cercano en eil.}
\end{figure}

\begin{figure}[H] 
  \centering
  \includegraphics[scale=0.5]{img/dist-vecino-ulysses.png}
  \caption{Camino generado por la heurística del vecino más cercano en ulysses.}
\end{figure}

\begin{figure}[H]
  \centering
  \includegraphics[scale=0.5]{img/dist-insercion-bayg.png}
  \caption{Camino generado por la heurística de la inserción más económica en bayg.}
\end{figure}

\begin{figure}[H]
  \centering
  \includegraphics[scale=0.5]{img/dist-insercion-eil.png}
  \caption{Camino generado por la heurística de la inserción más económica en eil.}
\end{figure}

\begin{figure}[H]
  \centering
  \includegraphics[scale=0.5]{img/dist-insercion-ulysses.png}
  \caption{Camino generado por la heurística de la inserción más económica en ulysses.}
\end{figure}

\begin{figure}[H]
  \centering
  \includegraphics[scale=0.5]{img/dist-kr-bayg.png}
  \caption{Camino generado por la heurística propia en bayg.}
\end{figure}

\begin{figure}[H]
  \centering
  \includegraphics[scale=0.5]{img/dist-kr-eil.png}
  \caption{Camino generado por la heurística propia en eil.}
\end{figure}

\begin{figure}[H]
  \centering
  \includegraphics[scale=0.5]{img/dist-kr-ulysses.png}
  \caption{Camino generado por la heurística propia en ulysses.}
\end{figure} 

A continuación se adjuntan también las representaciones gráficas de cada
uno de los caminos de que realizan los algoritmos.

\begin{figure}[H]
  \centering
  \includegraphics[scale=0.5]{../src/vecino_bayg.pdf}
  \caption{Representación gráfica del camino generado por la heurística del vecino más cercano en bayg.}
\end{figure} 

\begin{figure}[H]
  \centering
  \includegraphics[scale=0.5]{../src/vecino_eil.pdf}
  \caption{Representación gráfica del camino generado por la heurística del vecino más cercano en eil.}
\end{figure} 

\begin{figure}[H]
  \centering
  \includegraphics[scale=0.5]{../src/vecino_ulysses.pdf}
  \caption{Representación gráfica del camino generado por la heurística del vecino más cercano en ulysses.}
\end{figure} 

\begin{figure}[H]
  \centering
  \includegraphics[scale=0.5]{../src/insercion_bayg.pdf}
  \caption{Representación gráfica del camino generado por la heurística de inserción económica en bayg.}
\end{figure} 

\begin{figure}[H]
  \centering
  \includegraphics[scale=0.5]{../src/insercion_eil.pdf}
  \caption{Representación gráfica del camino generado por la heurística de inserción económica en eil.}
\end{figure}

\begin{figure}[H]
  \centering
  \includegraphics[scale=0.5]{../src/insercion_ulysses.pdf}
  \caption{Representación gráfica del camino generado por la heurística de inserción económica en ulysses.}
\end{figure} 

\begin{figure}[H]
  \centering
  \includegraphics[scale=0.5]{../src/kruskal_bayg.pdf}
  \caption{Representación gráfica del camino generado por la heurística $\alpha$-Kruskal en bayg.}
\end{figure} 
    \newpage

    \section{Conclusión}
    En este trabajo hemos implementado diversos algoritmos aplicando
la técnica Divide y Vencerás, y lo hemos enfrentado frente a soluciones aplicando
algoritmos clásicos mediante un análisis de eficiencia tanto teórico como empírico 
(incluso híbrido). 

Tras analizar los datos obtenidos de cada situación que se nos ha presentado,
hemos \textbf{extraído las siguientes conclusiones: }

\begin{itemize}
    \item La determinación del umbral óptimo es muy importante para garantizar un buen funcionamiento
    de la técnica Divide y Vencerás, \textbf{evitando un exceso de llamadas recursivas}.
    % \item El número de repeticiones de un algoritmo para la obtención de datos \textbf{depende del tamaño del 
    % del problema}, teniendo en cuenta la \textbf{casuística} de los datos.  
    \item En los equipos empleados, el orden de eficiencia es un factor limitante del mayor tamaño de 
    problema que puede llegar a ejecutase para un algoritmo determinado.
    \item La técnica Divide y Vencerás \textbf{no siempre proporciona una mejora de eficiencia}
    respecto al caso obvio. 
    
\end{itemize}
    \newpage

    \newpage
    \begin{thebibliography}{0}
        \bibitem{Verdegay2017} Verdegay Galdeano. (2017). Lecciones de Algorítmica / José Luis Verdegay. Técnica Avicam.
        \bibitem{Cormen2017} Cormen. (2017). Introduction to algorithms / Thomas H. Cormen... [et al.] (3rd ed.). PHI Learning.
        \bibitem{Garrido2018} Garrido Carrillo. (2018). Estructuras de datos avanzadas: con soluciones en C++ / A. Garrido. Universidad de Granada.  
        \bibitem{Rojo2022} Hu Chen. (2022). Práctica 1: Análisis de Eficiencia de Algoritmos / Shao Jie Hu, J. Samuel García y Mario Megías.      
    \end{thebibliography}
\end{document}